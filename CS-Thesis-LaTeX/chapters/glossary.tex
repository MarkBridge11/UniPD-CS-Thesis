\chapter{Glossario}
\label{cap:glossario}

\subsubsection*{Annotazione}
È un tipo di metadato che viene utilizzato per fornire informazioni aggiuntive su classi, metodi o campi all'interno di un'applicazione Spring. 
\subsubsection*{API}
Acronimo di Application Programming Interface, è un insieme di regole e protocolli che permettono a diversi componenti di comunicare tra loro, definendo il loro modo di interagire.
\subsubsection*{Audit}
Per traccia di audit (audit trail in inglese) si intende la registrazione dettagliata di attività o eventi in un database. Una traccia di audit contiene molteplici audit, cioè delle registrazioni di attività singole che possono contenere: data e ora, utente, operazioni e dati cambiati.
\subsubsection*{Business Intelligence}
Ci si riferisce a un insieme di processi aziendali o di tecnologie per raccogliere dati ed analizzare informazioni strategiche.
\subsubsection*{Business Logic}
In ambito dello sviluppo software per logica di business (business logic) si intende la logica che rende operativa un'applicazione. Ci si riferisce ad operazioni complesse ed elaborazioni sui dati.
\subsubsection*{CRUD}
Acronimo che sta per operazioni di base effettuate sui dati in un'applicazione o nel contesto informatico. Le operazioni a cui si fa riferimento sono: Create, Read, Update, Delete.
\subsubsection*{Class level}
Si utilizza questo termine per indicare che un elemento è condiviso da tutte le istanze di quella class e che opera allo stesso livello della classe.
\subsubsection*{Dialetto del server SQL}
In informatica, un "dialetto SQL" si riferisce ad una specifica implementazione del linguaggio SQL.
\subsubsection*{Dipendenza}
Pacchetti necessari alla costruzione del progetto.
\subsubsection*{Endpoint}
È un luogo digitale esposto tramite l'API dal quale l'API riceve le richieste e invia le risposte. Ogni endpoint è un URL che fornisce la posizione di una risorsa sul server.
\subsubsection*{Framework}
È una struttura dove un software può essere progettato e realizzato, facilitandone lo sviluppo al programmatore. Sono progettati per affrontare problemi comuni e offrire una base solida.
\subsubsection*{HTTP}
Acronimo di Hypertext Transfer Protocol, è un protocollo usato come sistema di trasmissione di informazioni sul web.
\subsubsection*{ICT}
Per ICT (tecnologie dell'informazione e della comunicazione, in italiano) si intende tutte le attività e le tecniche di elaborazione automatica, trasmissione e di archiviazione delle informazioni tramite computer ed elaboratori elettronici.
\subsubsection*{Mapping}
Per mapping si intende il processo di associazione (mappatura) dei dati tra due insiemi di dati.
\subsubsection*{Mock}
Con il termine mock o mockare o mockato, si intende la simulazione del comportamento di un componente. Termine utilizzato nel testing software poichè consente di isolare unità di codice e verificare che il software funzioni correttamente.
\subsubsection*{ORM - Object relational mapping}
È una tecnica di programmazione che facilita l'integrazione di sistemi software basati sulla programmazione orientata agli oggetti con sistemi di gestione di basi di dati relazionali.
\subsubsection*{Paginazione}
È una tecnica utilizzata quando siamo di fronte ad grande insieme di dati e permette di presentare all'utente il risultato in pagine di elementi. I parametri della paginazione possono essere: numero di pagina, elementi per pagina, direzione del risultato e ordinamento per termine.
\subsubsection*{Package}
Permettono di organizzare classi Java in gruppi logici in base al contesto. Essi possono contenere ulteriori package, i subpackage.
\subsubsection*{Parametri di query}
Sono dati o valori inseriti in una richiesta HTTP e vengono utilizzati per filtrare, ordinare o personalizzare i risultati. Nelle richieste HTTP i parametri di query vengono aggiunti dopo il "?" e separati da un "\&". Un esempio può essere l'URL: https://localohost:porta/richiesta/cerca?parametro1=prova\&parametro2=test.
\subsubsection*{Sandboxing}
Termine utilizzato nella sicurezza informatica per far eseguire applicazioni in un ambiente controllato, fornendo un set di risorse limitate.
\subsubsection*{Stakeholder}
Rappresentano i portatori di interesse, un insieme di persone coinvolte in decisioni importanti nel ciclo di vita di un software e che hanno influenza sul prodotto o sul processo.
\subsubsection*{Timestamp}
Si intende il formato standard di timestamp composto da data e ora con precisione al secondo.
\subsubsection*{Trigger}
Sono delle procedure associate ad una tabella di un database che si attivano automaticamente dopo un determinato evento su quella tabella. I possibili eventi sono operazioni di INSERT, UPDATE e DELETE.
\subsubsection*{UML}
Acronimo di Unified Modeling Language, creato per realizzazre un linguaggio di modellazione visivo comune.
\subsubsection*{URI}
Acronimo di Uniform Resource Identifier e consiste in una sequenza di caratteri che identifica una risorsa univocamente. Un esempio di URI sono gli URL.
\subsubsection*{URL}
Acronimo di Uniform Resource Locator che identifica univocamente una risorsa in rete.
\subsubsection*{UUID}
Acronimo di Universally unique identifier che permette di garantire l'univocità di una chiave. Esso è composto da 16 byte (128 bit). Nella sua forma canonica è rappresentato da 32 caratteri esadecimali, visualizzati in cinque gruppi separati da trattini, nella forma 8-4-4-4-12.