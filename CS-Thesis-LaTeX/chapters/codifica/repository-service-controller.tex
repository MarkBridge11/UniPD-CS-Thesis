\section{Repository-Service-Controller}

\begin{figure}[H] 
    \centering 
    \includegraphics[width=0.90\columnwidth]{rep-serv-contr-custom} 
    \caption{Schema di collegamento tra i componenti Controller, Service e Repository}
\end{figure}

\noindent In un'applicazione Spring Boot che implementa una REST API l'interazione tra questi tre componenti è fondamentale per gestire le richieste HTTP, elaborare la business logic e interagire con il database.\\
Come descritto nell'immagine l'interazione tra i componenti segue questo flusso:
\begin{enumerate}
\item il client invia una richiesta HTTP e arriva al Controller tramite URL;
\item il controller estrae i dati della richiesta, effettua controlli di validazione e chiama il metodo del service appropriato;
\item il metodo del service invocato conterrà la business logic per eseguire la richiesta e ulteriori controlli di validazione;
\item la/le repository coinvolta/e ottengono i dati richiesti dal database che vengono restituiti al service che li rielaborerà, restituitendo il risultato al controller;
\item il controller crea una risposta appropriata in formato JSON e la restituisce al client.
\end{enumerate}

\subsection{Controller}
\begin{figure}[H] 
    \centering 
    \includegraphics[width=0.4\columnwidth]{controllers} 
    \caption{Controller sviluppati}
\end{figure}
Il Controller è punto di instradamento per le richieste HTTP. Riceve le richieste dal client, estrae i dati dai parametri della richiesta o dal corpo (body), esegue appropriati controlli di validazione ed infine inoltra le operazioni al Service. I controlli di validazione comprendono: validazione degli ID, validazione del formato delle date inserite, validazione che un dato obbligatorio non sia nullo e controllo che i filtri inseriti non siano non valorizzati.\\
Ogni Controller è formato dalla sua interfaccia, contenente la firma dei metodi associati agli endpoint, e dalla sua implementazione. Ognuno di essi è annotato con \textit{@RestController}.

\subsubsection{Generazione automatica dello Swagger}
Per creare la documentazione dell'API è stata utilizzata la dipendenza \textit{springdoc-openapi-starter-webmvc-ui} citata nella sotto-sezione \hyperlink{doc-api}{4.7.2.1} nel capitolo di Progettazione. Grazie a questa dipendenza è stato possibile generare automaticamente la documentazione dell'API basata sulle annotazioni presenti nel codice sorgente. SpringDoc API utilizza le annotazioni di Spring per identificare i Controller, i metodi delle API, i dati in input e in output. Per visualizzare questo documento generato è possibile accederci tramite un endpoint specifico: \textit{https://localhost:porta\_scelta/swagger-ui.html}.
\begin{figure}[H] 
    \centering 
    \includegraphics[width=0.90\columnwidth]{api-doc-annotazioni} 
    \caption{Annotazioni utili a generare la documentazione per l'API}
\end{figure}
\noindent La dipendenza importata fornisce anche annotazioni per arrichire il documento. In questa immagine notiamo le seguenti annotazioni:
\begin{itemize}
\item \textit{@Operation}, specifica un sommario dell'endpoint che l'utente può vedere prima di visualizzare nella sua totalità l'endpoint e un campo descrizione per fornire una descrizione più approfondita sul funzionamento;
\item \textit{@ApiResponses}, formata da tanti \textit{@ApiResponse} che specificano il codice di errore, la descrizione di quando questo errore esce e lo schema che avrà quando si presenterà.
\end{itemize}
Per agevolare la lettura dell'API è stata anche utilizzata la notazione \textit{@Schema} per fornire un nome appropriato a campi o classi.









\begin{figure}[H] 
    \centering 
    \includegraphics[width=0.9\columnwidth]{controller-interfaccia} 
    \caption{Firma di un metodo nell'interfaccia di un Controller}
\end{figure}
\noindent All'interno dell'interfaccia del Controller troviamo tutti i metodi associati agli endpoint. Per identificare a quale endpoint appartiene vengono utilizzate annotazioni come: \textit{@PostMapping}, \textit{@GetMapping}, \textit{@DeleteMapping}, \textit{@PutMapping} e \textit{@PatchMapping}.\\
Tramite l'annotazione \textit{@RequestBody} Spring deserializza\textsubscript{g} automaticamente il JSON in un tipo Java, in questo caso \texttt{MilestoneListRequest}.
Come libreria standard utilizzata per mappare i campi JSON inseriti nella request dall'utente in un oggetto Java, viene utilizzata la libreria Jackson\textsubscript{g}.

\begin{figure}[H] 
    \centering 
    \includegraphics[width=0.8\columnwidth]{controller-validazione} 
    \caption{Esempio di controllo di validazione}
\end{figure}
\noindent Nelle classi \texttt{ControllerImpl} troviamo l'implementazione dei metodi dell'interfaccia, controlli di validazione che possono lanciare eccezioni e il metodo del Service utilizzato.\\
In questo code snippet vediamo il controllo di validazione su l'Id del Progetto associato ad una Milestone. Se passerà il controllo, allora verrà inoltrata la richiesta al Service.

\subsection{Repository}
\begin{figure}[H] 
    \centering 
    \includegraphics[width=0.4\columnwidth]{repositories} 
    \caption{Repository sviluppati}
\end{figure}
Un Repository in Spring Boot fornisce un modo per interagire con una fonte di dati, in questo caso un database. Permette di eseguire operazioni CRUD tramite Spring Data JPA che genera automaticamente query SQL in base ai metodi che l'interfaccia del Repository contiene.\\
Le Repository da me create estendono, per poter eseguire le query, l'interfaccia \texttt{JpaRepository} o \texttt{ PagingAndSortingRepository} per risposte paginate.\\
Ogni Repository deve essere annotato con \textit{@Repository}. 
\begin{figure}[H] 
    \centering 
    \includegraphics[width=0.9\columnwidth]{repository-example} 
    \caption{Esempio di una Repository}
\end{figure}
\noindent È possibile creare delle query personalizzate SQL tramite parole chiave di Spring Data JPA. In questo caso nell'immagine la query corrisponderà alla ricerca della Pianificazione che ha la stessa Risorsa e lo stesso Ruolo.

\subsection{Service}
\begin{figure}[H] 
    \centering 
    \includegraphics[width=0.4\columnwidth]{services} 
    \caption{Service sviluppati}
\end{figure}
Il Service è un componente che contiene elaborazioni di dati, operazioni complesse, chiamate a metodi nei Repository per accedere al database e restituisce i risultati al Controller.\\
Ogni Service è annotato \textit{@Service}. Può contenere fino a tre casistiche diverse di logiche implementative che sono elencate nelle prossime sotto-sezioni.

\subsubsection{Esempio con utilizzo di metodi dei Repository}
Questa casistica è la più frequente poiché permette di effettuare le semplici operazioni CRUD che gli endpoint solitamente richiedono. In questi casi vengono utilizzati metodi dei Repository come: \texttt{save()} per creare una nuova istanza di un'entità nel database, \texttt{deleteById()} per cancellare un'istanza da database e \texttt{findById()} per ricercare un'istanza precisa di un'entità nel database.
\begin{figure}[H] 
    \centering 
    \includegraphics[width=0.95\columnwidth]{service-delete-example} 
    \caption{Metodo deleteMilestone() nel Service}
\end{figure}
\noindent In questo code snippet possiamo osservare come funziona il metodo \texttt{deleteMilestone()} all'interno del \texttt{MilestoneService}.Viene eseguito un controllo tramite \texttt{findById()} per verificare se l'ID è presente nel database e in caso negativo viene lanciata un'eccezione che impedisce l'eliminazione della Milestone. Questa eccezione comporta la restituzione di un codice di errore 404 per informare l'utente che la Milestone con l'ID specificato non esiste nel database e quindi non può essere cancellata.\\
Il controllo successivo verifica che la Milestone non sia associata ad una Pianificazione ed in caso di esito positivo lancia un'eccezione che contiene un codice di errore 409.\\
Se tutti questi controlli non interrompono l'esecuzione, allora avverrà la cancellazione nel database tramite il metodo \texttt{deleteById()} della Milestone.\\
Viene infine generata la response da ritornare al Controller formata da un oggetto "metadata" vuoto, dato che non ci sono stati errori, e una nuova \texttt{MilestoneResponse}, contenente la Milestone rimossa che comporrà l'oggetto "data".

\subsubsection{Esempio di creazione query con filtri}
Un'altra logica rilevante utilizzata nei Service è quella che server a generare una query dinamica in base ai filtri inseriti dall'utente.\\
Dato che la query con Spring Data JPA sarebbe stata troppo lunga e poco leggibile, è stato deciso di crearla tramite \textit{CriteriaBuilder}\textsubscript{g} e \textit{CriteriaQuery}\textsubscript{g}, che permettono di creare query dinamiche in modo programmatico.
\begin{figure}[H] 
    \centering 
    \includegraphics[width=0.90\columnwidth]{criteriaquery} 
    \caption{Code snippet della creazione della query basata su filtri inseriti}
\end{figure}
\noindent Nel seguente estratto di codice ho ridotto la complessità mantenendo solo le parti essenziali che illustrano il funzionamento.\\
Nel primo blocco di istruzioni inizializziamo \textit{CriteriaBuilder}, una nuova \textit{CriteriaQuery} e selezioniamo tutte le \texttt{Milestone} uniche.\\
Nelle successive istruzioni andiamo a creare tutte le operazioni di \textit{Join<>} tra le entità, inseriamo dei predicati (condizioni logiche, utili quando la query è dinamica) nella clausola \texttt{where()} della query e infine generiamo la query simulando la paginazione, recuperando la lista di risultati in una lista di \texttt{Milestone}.

\subsubsection{Esempio di creazione file Excel}
\begin{figure}[H] 
    \centering 
    \includegraphics[width=0.95\columnwidth]{create-excel-method} 
    \caption{Code snippet semplificato della creazione di un file Excel}
\end{figure}
L'ultima logica rilevante è quella utilizzata per la creazione di un file Excel. In questo frammento di codice è stato semplificato il funzionamento per mostrare i punti cardine.\\
La sintassi utilizzata dalla libreria Apache POI è molto semplice e intuitiva. Nelle prime righe andiamo a creare una nuova riga \texttt{Row} per l'header e una che conterrà valori. Per ogni riga bisognerà creare poi una cella \texttt{Cell}. Successivamente è possibile applicare uno stile alla cella \texttt{CellStyle} e il suo valore.\\
Dopo che il file Excel è stato compilato, il contenuto verrà scritto su un array di byte di tipo \texttt{ByteArrayOutputStream}.
Questo array verrà restituito al Controller che specificherà nella risposta HTTP fornita:
\begin{enumerate}
\item l'intestazione "Content-Disposition" nel formato "attachment", che indica al browser che il contenuto dovrebbe essere trattato come un allegato scaricabile;
\item imposta il tipo di contenuto del file, in questo caso Excel con estensione .xls;
\item specifica nel body della risposta l'array ritornato dal Service.
\end{enumerate}
Il client vedrà la finestra di dialogo di download nel browser e potrà scaricare il file.
\begin{figure}[H] 
    \centering 
    \includegraphics[width=1.00\columnwidth]{risposta-excel} 
    \caption{Response fornita dal Controller contenente il file Excel da scaricare}
\end{figure}