\section{Package dto}
Per la composizione delle richieste che il client invia o delle risposte che il server manda, sono state create vari tipi di requests e responses.\\
Ogni oggetto di risposta è composto da un oggetto "data" e un oggetto "metadata", mentre gli oggetti di richiesta possono variare. Se la richiesta è costruita per un endpoint che restituisce una lista di risultati, possiamo avere due casistiche:
\begin{itemize}
\item corpo della richiesta (body) formato da un oggetto "data", contenente informazioni utili alla risposta, e "metadata" che contiene i dati di paginazione;
\item nessun corpo, ma soltanto i dati di paginazione passati come parametri di query\textsubscript{g}.
\end{itemize}
Se invece la richiesta è costruita per un endpoint che restituisce un singolo oggetto nell'oggetto data, non viene utilizzato alcun body o parametri di query, ma solo una path variable.\\
Questa suddivisione "data" e "metadata" è un modello comune nella progettazione API per separare i dati effettivi o dati che forniscono informazioni per ottenere un tipo di risposta, dalle informazioni aggiuntive che descrivono il risultato o aggiungono caratteristiche alla richiesta.\\
Questo package è formato da vari subpackage.
\subsection{Common}
\begin{figure}[H] 
    \centering 
    \includegraphics[width=1.0\columnwidth]{dto-sub-common} 
    \caption{Subpackage common}
\end{figure}
Questo subpackage contiene la response di base, response paginata, la response utilizzata nell'eccezione personalizzata, una request per la paginazione e i vari metadata utilizzati nelle responses.\\ Ogni \texttt{Metadata} contiene un boolean che indica se si è andati in errore (true) o meno (false), il messaggio di errore e una lista di \texttt{Validation}, contenente informazioni di validazione dei dati.\\
Ogni \texttt{Response} verrà poi estesa dalle classi che costituiscono le risposte ritornate per entità nel subpackage Responses.\\
È stata inoltre creata una classe \texttt{Pagination} per avere un controllo personalizzato sulla Paginazione.
\subsection{Requests e Body}
\begin{figure}[H] 
    \centering 
    \includegraphics[width=1.0\columnwidth]{dto-sub-requests-body-2} 
    \caption{Subpackage requests e body}
\end{figure}
Questi due subpackages sono correlati tra di loro dato che ogni body ha come attributo una request presa dal subpackage requests.
Esistono due tipi di body:
\begin{itemize}
\item ogni \texttt{Request} singola è utilizzata per richiedere una specifica risorsa e contiene una request DTO formata dai campi utili che l'utente deve inserire;
\item ogni \texttt{ListRequest} è utilizzata per richiedere una o più risorse ed è formata da un oggetto data di tipo \texttt{ListRequestInternal}, che contiene tre parametri:
\begin{itemize}
\item filters, lista di oggetti \texttt{Filter} formati da campo-valore;
\item una stringa q (quicksearch) utile nella query personalizzata per cercare in determinati campi quello che l'utente inserisce;
\item andOperator per impostare i filtri in And o in Or nella query.
\end{itemize} 
\end{itemize}
All'interno del subpackage requests troviamo anche \texttt{PatchRequest}, contenente un unico campo corrispondente ad una lista di \texttt{PatchField}. Questa richiesta viene utilizzata nella richieste PATCH, in cui è possibile inserire uno o più valori in base a cosa si va a modificare.

\subsection{Responses}
\begin{figure}[H] 
    \centering 
    \includegraphics[width=0.9\columnwidth]{dto-sub-responses} 
    \caption{Subpackage responses e dtoresponses}
\end{figure}
Il package responses contiene tutte le response per entità. Ogni response estende la classe astratta \texttt{Response} se è una risposta singola, altrimenti estende la classe \texttt{PaginatedResponse} se è una lista di risposte. Questo permette all'utente di visualizzare la risposta, dopo aver interagito con un'endpoint, formata da "data" e "metadata".\\
Ogni response contiene un solo attributo che corrisponde all'oggetto "data" e può essere:
\begin{itemize}
\item una response DTO del subpackage dtoresponses;
\item un'entità DTO del subpackage delle entità;
\item in caso di una \texttt{ListResponse} l'attributo sarà una lista di response.
\end{itemize}