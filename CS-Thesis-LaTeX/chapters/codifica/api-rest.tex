\section{Endpoint sviluppati dell'API REST}
In questa sezione si trovano le descrizioni di tutti gli endpoint implementati, suddivisi in base all'ambito di interesse del servizio. Sarà fornito anche il verbo HTTP e il percorso necessario per effettuare ciascuna richiesta.\\
I verbi standard forniti da HTTP utilizzati applicati all'API REST sono i seguenti:
 
\setlength{\arrayrulewidth}{0.3mm}
\renewcommand{\arraystretch}{2.5}
\begin{center}
\rowcolors{1}{white}{mygray}
\begin{longtable}{p{2cm}|p{8cm}}
\textbf{Verbo}  & \textbf{Descrizione}\\
\hline
%\rowcolor{mygray} 
POST   & Utilizzato per inviare dati al server al fine di creare una nuova risorsa ed aggiungerla all'insieme corrente\\
GET    & Utilizzato per richiedere dati al server in merito ad una o più risorse senza modificarle          \\
%\rowcolor{mygray}
DELETE &   Utilizzato per eliminare una risorsa specifica dal server          \\
PUT    &   Utilizzato per aggiornare totalmente una risorsa e viene utilizzato quando si vuole sostituire completamente una risorsa          \\
%\rowcolor{mygray}
PATCH  &     Utilizzato per effettuare aggiornamenti parziali a una risorsa esistente        \\ 
\hline
\hiderowcolors
\caption{Verbi Standard HTTP utilizzati}
\label{tab:verbi-http}
\end{longtable}
\end{center}

%Per garantire una lettura chiara dello stato dell'operazione richiesta, sono stati utilizzati i seguenti codici di risposta HTTP. Questi codici 

\noindent Di seguito elenco gli endpoint organizzati per ambito di interesse, specificato il verbo HTTP utilizzato, il path per effettuare la richiesta e una descrizione della sua funzionalità.
\subsection*{Anagrafiche}
Questi endpoint vengono utilizzati dal Program Manager per effettuare operazioni di lettura sulle dati aziendali. 
\setlength{\arrayrulewidth}{0.3mm}
\renewcommand{\arraystretch}{2.5}
\begin{center}
\rowcolors{1}{white}{mygray}
\begin{longtable}{p{1.3cm}|p{4.95cm}|p{5.7cm}}
\textbf{Verbo}  & \textbf{Path} & \textbf{Descrizione}\\
\hline
GET    & \texttt{/area-competenza/\{id\}} & Permette la lettura di un'Area di Compentenza dato l'ID\\
GET    & \texttt{/clienti} & Permette la lettura da DB di tutti i Clienti ottenendo come risposta la lista di tutti i Clienti\\
GET    & \texttt{/clienti/\{id\}} & Permette la lettura da DB di un Cliente dato l'Id ottenendo come risposta il Cliente corrispondente\\
GET    & \texttt{/risorse/\{id\}} & Permette la lettura da DB di un Risorsa dato l'Id ottenendo come risposta il Risorsa corrispondente\\
GET    & \texttt{/ruoli} & Permette la lettura da DB di tutti i Ruoli ottenendo come risposta la lista di tutti i Ruoli\\
GET    & \texttt{/ruoli/\{id\}} & Permette la lettura da DB di un Ruolo dato l'Id ottenendo come risposta il Ruolo corrispondente\\
GET    & \texttt{/skill} & Permette la lettura da DB di tutte le Skill ottenendo come risposta la lista di tutte le Skill\\
GET    & \texttt{/skill/\{id\}} & Permette la lettura da DB di una Skill dato l'Id ottenendo come risposta la Skill corrispondente\\
\hline
\hiderowcolors
\caption{Endpoint Anagrafiche sviluppati}
\label{tab:endpoint-anagrafiche-api}
\end{longtable}
\end{center}

\subsection*{Milestones}
Questi endpoint consentono la creazione, lettura, modifica ed eliminazione di Milestones. Vengono utilizzati principalmente dal Program Manager.
\setlength{\arrayrulewidth}{0.3mm}
\renewcommand{\arraystretch}{2.5}
\begin{center}
\rowcolors{1}{white}{mygray}
\begin{longtable}{p{1.3cm}|p{4.95cm}|p{5.7cm}}
\textbf{Verbo}  & \textbf{Path} & \textbf{Descrizione}\\
\hline
POST    & \texttt{/milestones} & Permette l'inserimento di una nuova Milestone, inserendo il Progetto associato, il Commerciale, la Pianificazione e altri dettagli.\\
POST    & \texttt{/milestones/list} & Rappresenta una POST di ricerca. Restituisce una lista di Milestone in base a: filtri inseriti dall'utente, parola chiave nella quicksearch (q), booleano in cui si decide se applicare tutti i filtri (AND) o che ne valga almeno uno (OR).\\
PATCH    & \texttt{/milestones/\{id\}/type} & Permette la modifica del campo Type di una Milestone su DB.\\
GET    & \texttt{/milestones/\{id\}} & Permette la lettura da DB di una Milestone dato l'Id ottenendo come risposta la Milestone corrispondente.\\
DELETE    & \texttt{/milestones/\{id\}} & Permette l'eliminazione da DB di una Milestone dato l'Id ottenendo come risposta la Milestone eliminata.\\
\hline
\hiderowcolors
\caption{Endpoint Milestones sviluppati}
\label{tab:endpoint-milestones-api}
\end{longtable}
\end{center}

\subsection*{Pianificazioni}
Questi endpoint consentono la creazione, lettura, modifica ed eliminazione di Pianificazioni.
Vengono utilizzati principalmente dal Program Manager.
\setlength{\arrayrulewidth}{0.3mm}
\renewcommand{\arraystretch}{2.5}
\begin{center}
\rowcolors{1}{white}{mygray}
\begin{longtable}{p{1.3cm}|p{4.95cm}|p{5.7cm}}
\textbf{Verbo}  & \textbf{Path} & \textbf{Descrizione}\\
\hline
GET    & \texttt{/pianificazioni/\{id\}} & Permette la lettura da DB di una Pianificazione dato l'Id ottenendo come risposta la Pianificazione corrispondente.\\
PUT & \texttt{/pianificazioni/\{id\}} & Permette la modifica parziale o totale di campi di una Pianificazione su DB sovrascrivendola.\\
DELETE    & \texttt{/pianificazioni/\{id\}} & Permette l'eliminazione da DB di una Pianificazione dato l'Id ottenendo come risposta la Pianificazione eliminata.\\
POST    & \texttt{/pianificazioni} & Permette l'inserimento di una nuova Pianificazione ed eventuali entità associate come: risorsa, ruolo, milestone, progetto, responsabile e figura.\\
POST    & \texttt/{pianificazioni/xlsx} & Permette l'esportazione di report Excel di Pianificazioni in base a filtri inseriti dall'utente.\\
POST    & \texttt{/pianificazioni/list} & Rappresenta una POST di ricerca. Restituisce una lista di Pianificazioni in base a: filtri inseriti dall'utente, parola chiave nella quicksearch (q), booleano in cui si decide se applicare tutti i filtri (AND) o che ne valga almeno uno(OR).\\
PATCH    & \texttt{/pianificazioni/\{id\}/times} & Permette la modifica dei campi Data Inizio e Data Fine di una Pianificazione su DB.\\
PATCH    & \texttt{/pianificazioni/\{id\}/festivi} & Permette la modifica del campo Festivi di una Pianificazione su DB.\\
GET    & \texttt{/pianificazioni/\{id\}/audit} & Permette la lettura delle modifiche effettuate su una Pianificazione dalla tabella di log.\\
\hline
\hiderowcolors
\caption{Endpoint Pianificazioni sviluppati}
\label{tab:endpoint-pianificazioni-api}
\end{longtable}
\end{center}


\subsection*{Richieste}
Questi endpoint consetono la creazione, lettura, modifica ed eliminazione di Richieste di Pianificazioni. Vengono utilizzati principalmente dal Project Manager.
\setlength{\arrayrulewidth}{0.3mm}
\renewcommand{\arraystretch}{2.5}
\begin{center}
\rowcolors{1}{white}{mygray}
\begin{longtable}{p{1.3cm}|p{4.95cm}|p{5.7cm}}
\textbf{Verbo}  & \textbf{Path} & \textbf{Descrizione}\\
\hline
GET    & \texttt{/richieste/\{id\}} & Permette la lettura da DB di una Richiesta dato l'Id ottenendo come risposta la Richiesta corrispondente.\\
PUT    & \texttt{/richieste/\{id\}} & Permette la modifica parziale o totale di campi di una Richiesta su DB sovrascrivendola.\\
DELETE    & \texttt{/richieste/\{id\}} & Permette l'eliminazione da DB di una Richiesta dato l'Id ottenendo come risposta la Richiesta eliminata.\\
POST    & \texttt{/richieste} & Permette l'inserimento di una nuova Richiesta, inserendo le Figure Professionali e le Skill richieste e vari dettagli.\\
POST    & \texttt{/richieste/list} & Rappresenta una POST di ricerca. Restituisce una lista di Richieste in base a: filtri inseriti dall'utente, parola chiave nella quicksearch (q), booleano in cui si decide se applicare tutti i filtri (AND) o che ne valga almeno uno (OR)\\
DELETE    & \texttt{/richieste/\{id\}/stato} & Permette la modifica del campo Stato di una Richiesta su DB.\\
DELETE    & \texttt{/richieste/\{id\}/priorita} & Permette la modifica del campo Priorita di una Richiesta su DB.\\
\hline
\hiderowcolors
\caption{Endpoint Pianificazioni sviluppati}
\label{tab:endpoint-richieste-api}
\end{longtable}
\end{center}

