\section{Swagger}
%Spiegazione come è stata creata la grafica per gli endpoint con screen di un'interfaccia Controller.

Swagger è un framework utilizzato per la documentazione, progettazione e gestione di API REST.
Per garantire ordine sul mio lavoro e una comunicazione più trasparente e chiara con chi lavora nel lato Front-End è stato creato questo documento utilizzando lo standard OpenAPI.\\
Questo disegno dell'API ha sostituito poi il mock iniziale dell'API, considerando questo nuovo documento come unico per la consultazione dell'API finale.
\begin{figure}[H] 
    \centering 
    \includegraphics[width=0.95\columnwidth]{foto-swagger-titolo} 
    \caption{Inizio del documento di definizione dell'API}
\end{figure}
\noindent Il documento rappresenta un disegno completo dell'implementazione dell'API REST. Contiene tutti gli endpoint sviluppati, come sono strutturati al loro interno e arrichito di descrizioni per endpoint, risposte, richieste ed errori.
\begin{figure}[H] 
    \centering 
    \includegraphics[width=0.95\columnwidth]{foto-swagger-example-value-2} 
    \caption{Example value del body richiesto dall'endpoint}
\end{figure}
\noindent In questa immagine possiamo notare un "Example value" del body richiesto da parte dell'endpoint POST /pianificazioni.\\
\begin{figure}[H] 
    \centering 
    \includegraphics[width=0.65\columnwidth]{foto-swagger-composizione-oggetto} 
    \caption{Composizione oggetto response ritornato dall'endpoint}
\end{figure}
\noindent Inoltre, è possibile esaminare la struttura di ciascun oggetto, sia nelle richieste che nelle risposte. Questo è particolarmente vantaggioso per gli sviluppatori Front-End, poiché consente loro di interagire in modo accurato con il Back-End. In questo modo, il Back-End indica al Front-End cosa e come ci si aspetta di ricevere per quanto riguarda i dati nelle richieste, mentre il Front-End viene informato su come e cosa riceverà dal Back-End nelle risposte.
\begin{figure}[H] 
    \centering 
    \includegraphics[width=1.00\columnwidth]{foto-swagger-errori} 
    \caption{Errori possibili in un endpoint}
\end{figure}
\noindent Per garantire una comprensione chiara dei possibili errori che potrebbero verificarsi, vengono identificate e descritte varie situazioni, ciascuna associata a un codice di errore specifico, al fine di fornire una struttura generale per codice di errore.
