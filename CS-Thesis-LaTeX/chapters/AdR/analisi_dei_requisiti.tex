\chapter{Analisi dei Requisiti}
\label{cap:analisi-requisiti}

Il seguente capitolo di Analisi dei Requisiti rappresenta una dettagliata e approfondita esplorazione delle necessità e delle aspettative che guidano la creazione e lo sviluppo del progetto in questione. Questa analisi è stata condotta al fine di definire chiaramente gli obiettivi e le funzionalità del prodotto, fornendo una base solida per la progettazione e l'implementazione del software.\\

\section{Descrizione generale}
Ogni caso d'uso è stato schematizzato secondo i seguenti punti:
\begin{itemize}
\item \textbf{attore coinvolto:} in cui si specifica l'attore;
\item \textbf{descrizione:} offre una spiegazione più dettagliata del caso d'uso; 
\item \textbf{precondizioni:} rappresenta la condizione che deve essere soddisfatta e verificata affinchè il caso d'uso possa essere eseguito con successo;
\item \textbf{postcondizioni:} rappresenta lo stato dell'attore in seguito all'esecuzione con successo del caso d'uso;
\item \textbf{estensioni:} in cui si specificano le eventuali estensioni collegate;
\item \textbf{inclusioni:} in cui si specificano le eventuali inclusioni.
\end{itemize}
Vengono inserite anche delle immagini dell'UML\textsubscript{g} per fornire una spiegazione visiva che può aiutare maggiormente la comprensione.

\section{Semplificazioni adottate nei casi d'uso}
All'interno dei casi d'uso è possibile leggere l'abbreviazione "vis." . Il seguente termine è utilizzato per abbreviare la parola "Visualizzazione".\\
Per agevolare la lettura delle immagini dei casi d'uso non è stato inserito il collegamento tra gli scenari principali e il database. È dato per scontato quindi, che ogni informazione venga recuperata dal database.\\

\section{Attori}
\begin{figure}[H] 
    \centering 
    \includegraphics[width=0.3\columnwidth]{usecase/attori} 
    \caption{Attori}
\end{figure}
Gli attori che possiamo trovare all'interno dei casi d'uso rappresentano due risorse aziendali:
\subsubsection*{Project Manager}
Conosciuto come il "Responsabile di Progetto", si occupa dell'avvio, pianificazione ,esecuzione e controllo di un singolo progetto, seguendo tecniche di project management.\\
Le sue principali responsabilità sono:
\begin{itemize}
\item assicurarsi che i progetti siano allineati con gli obiettivi aziendali;
\item coordinare le risorse umane, assicurandosi che vengano utilizzate in modo efficiente;
\item stabilisce milestone, scadenze e obiettivi, monitorando lo stato di avanzamento dei progetti;
\item comunica con stakeholder, team di progetto e altre parti interessate.
\end{itemize}
\subsubsection*{Program Manager}
È un ruolo di gestione all'interno di un'organizzazione, ed è responsabile della pianificazione complessiva e del controllo di più progetti che compongono il suo programma. Collabora strettamente col Project Manager ed i loro compiti spesso si sovrappongono ma differiscono di portata, in quanto il Program Manager supervisiona gruppi di progetti gestiti singolarmente dai Project Manager.

\section{Casi d'uso}

\subsection{Scenario Anagrafiche}
\begin{figure}[H] 
    \centering 
    \includegraphics[width=1.1\columnwidth]{usecase/anagrafiche-general} 
    \caption{Casi d'Uso del scenario Anagrafiche}
\end{figure}

\subsubsection*{UC1 - Vis. Risorsa}

\begin{figure}[H] 
    \centering 
    \includegraphics[width=0.6\columnwidth]{usecase/UC1} 
    \caption{Caso d'Uso 1 espanso}
\end{figure}

\begin{itemize}[label=$\circ$]
\item \textbf{Attore:} Program Manager;
\item \textbf{Descrizione:} il Program Manager può visualizzare una Risorsa;
\item \textbf{Precondizioni:} il richiedente è un Program Manager;
\item \textbf{Postcondizioni:} la Risorsa selezionata è visualizzabile dal Program Manager;
\item \textbf{Estensioni:} UC2;
\item \textbf{Inclusioni:} UC3, UC4, UC5.
\end{itemize}

\subsubsection*{UC1.1 - Vis. nome}
\begin{itemize}[label=$\circ$]
\item \textbf{Attore:} Program Manager;
\item \textbf{Descrizione:} il Program Manager può visualizzare il nome di una Risorsa;
\item \textbf{Precondizioni:} la Risorsa è visualizzabile dal Program Manager;
\item \textbf{Postcondizioni:} il Program Manager può visualizzare il nome della Risorsa selezionata;
\item \textbf{Estensioni:} il caso d'uso non ha estensioni;
\item \textbf{Inclusioni:} il caso d'uso non ha inclusioni.
\end{itemize}

\subsubsection*{UC1.2 - Vis. cognome}
\begin{itemize}[label=$\circ$]
\item \textbf{Attore:} Program Manager;
\item \textbf{Descrizione:} il Program Manager può visualizzare il cognome di una Risorsa;
\item \textbf{Precondizioni:}  la Risorsa è visualizzabile dal Program Manager;
\item \textbf{Postcondizioni:} il Program Manager può visualizzare il cognome della Risorsa selezionata;
\item \textbf{Estensioni:} il caso d'uso non ha estensioni;
\item \textbf{Inclusioni:} il caso d'uso non ha inclusioni.
\end{itemize}

\subsubsection*{UC1.3 - Vis. email}
\begin{itemize}[label=$\circ$]
\item \textbf{Attore:} Program Manager;
\item \textbf{Descrizione:} il Program Manager può visualizzare l'email di una Risorsa;
\item \textbf{Precondizioni:}  la Risorsa è visualizzabile dal Program Manager;
\item \textbf{Postcondizioni:} il Program Manager può visualizzare l'email della Risorsa selezionata;
\item \textbf{Estensioni:} il caso d'uso non ha estensioni;
\item \textbf{Inclusioni:} il caso d'uso non ha inclusioni.
\end{itemize}

\subsubsection*{UC2 - Vis. errore di validazione Anagrafiche}
\begin{itemize}[label=$\circ$]
\item \textbf{Attore:} Program Manager;
\item \textbf{Descrizione:} questo caso d'uso descrive anche UC3.2,UC4.2,UC5.2,UC6.2. Viene visualizzato un messaggio di errore in caso vengano eseguite funzionalità con dati non validi. Esso rappresenta errori di validazione nella richiesta fornita dall'utilizzatore;
\item \textbf{Precondizioni:} il Program Manager sta effettuando operazioni con dati non validi;
\item \textbf{Postcondizioni:} l'esecuzione della funzionalità è interrotta e viene visualizzato il messaggio di errore;
\item \textbf{Estensioni:} il caso d'uso non ha estensioni;
\item \textbf{Inclusioni:} il caso d'uso non ha inclusioni.
\end{itemize}

\subsubsection*{UC3 - Vis. lista Skill}
\begin{figure}[H] 
    \centering 
    \includegraphics[width=0.6\columnwidth]{usecase/UC3} 
    \caption{Caso d'Uso 3 espanso}
\end{figure}
\begin{itemize}[label=$\circ$]
\item \textbf{Attore:} Program Manager;
\item \textbf{Descrizione:} il Program Manager può visualizzare la lista delle Skill;
\item \textbf{Precondizioni:} il richiedente è un Program Manager;
\item \textbf{Postcondizioni:} la lista delle Skill è visualizzabile dal Program Manager;
\item \textbf{Estensioni:} il caso d'uso non ha estensioni;
\item \textbf{Inclusioni:} il caso d'uso non ha inclusioni.
\end{itemize}

\subsubsection*{UC3.1 - Vis. singola Skill}
\begin{figure}[H] 
    \centering 
    \includegraphics[width=0.6\columnwidth]{usecase/UC3.1} 
    \caption{Caso d'Uso 3.1 espanso}
\end{figure}
\begin{itemize}[label=$\circ$]
\item \textbf{Attore:} Program Manager;
\item \textbf{Descrizione:} il Program Manager può visualizzare una Skill;
\item \textbf{Precondizioni:} il richiedente è un Program Manager;
\item \textbf{Postcondizioni:} la Skill selezionata è visualizzabile dal Program Manager;
\item \textbf{Estensioni:} UC3.2;
\item \textbf{Inclusioni:} il caso d'uso non ha inclusioni.
\end{itemize}

\subsubsection*{UC3.1.1 - Vis. descrizione Skill}
\begin{itemize}[label=$\circ$]
\item \textbf{Attore:} Program Manager;
\item \textbf{Descrizione:} il Program Manager può visualizzare la descrizione di una Skill;
\item \textbf{Precondizioni:} la Skill è visualizzabile dal Program Manager;
\item \textbf{Postcondizioni:} il Program Manager può visualizzare la descrizione della Skill selezionata;
\item \textbf{Estensioni:} il caso d'uso non ha estensioni;
\item \textbf{Inclusioni:} il caso d'uso non ha inclusioni.
\end{itemize}

\subsubsection*{UC4 - Vis. lista Ruoli}
\begin{figure}[H] 
    \centering 
    \includegraphics[width=0.6\columnwidth]{usecase/UC4} 
    \caption{Caso d'Uso 4 espanso}
\end{figure}
\begin{itemize}[label=$\circ$]
\item \textbf{Attore:} Program Manager;
\item \textbf{Descrizione:} il Program Manager può visualizzare la lista dei Ruoli;
\item \textbf{Precondizioni:} il richiedente è un Program Manager;
\item \textbf{Postcondizioni:} la lista dei Ruoli è visualizzabile dal Program Manager;
\item \textbf{Estensioni:} il caso d'uso non ha estensioni;
\item \textbf{Inclusioni:} il caso d'uso non ha inclusioni.
\end{itemize}

\subsubsection*{UC4.1 - Vis. singolo Ruolo}
\begin{figure}[H] 
    \centering 
    \includegraphics[width=0.6\columnwidth]{usecase/UC4.1} 
    \caption{Caso d'Uso 4.1 espanso}
\end{figure}
\begin{itemize}[label=$\circ$]
\item \textbf{Attore:} Program Manager;
\item \textbf{Descrizione:} il Program Manager può visualizzare un Ruolo;
\item \textbf{Precondizioni:} il richiedente è un Program Manager;
\item \textbf{Postcondizioni:} il Ruolo selezionato è visualizzabile dal Program Manager;
\item \textbf{Estensioni:}  UC4.2;
\item \textbf{Inclusioni:} il caso d'uso non ha inclusioni.
\end{itemize}

\subsubsection*{UC4.1.1 - Vis. descrizione Ruolo}
\begin{itemize}[label=$\circ$]
\item \textbf{Attore:} Program Manager;
\item \textbf{Descrizione:} il Program Manager può visualizzare la descrizione di un Ruolo;
\item \textbf{Precondizioni:}  il Ruolo è visualizzabile dal Program Manager;
\item \textbf{Postcondizioni:} il Program Manager può visualizzare la descrizione del Ruolo selezionato;
\item \textbf{Estensioni:} il caso d'uso non ha estensioni;
\item \textbf{Inclusioni:} il caso d'uso non ha inclusioni.
\end{itemize}

\subsubsection*{UC5 - Vis. lista Area di Competenza}
\begin{figure}[H] 
    \centering 
    \includegraphics[width=0.6\columnwidth]{usecase/UC5} 
    \caption{Caso d'Uso 5 espanso}
\end{figure}
\begin{itemize}[label=$\circ$]
\item \textbf{Attore:} Program Manager;
\item \textbf{Descrizione:} il Program Manager può visualizzare la lista delle Aree di Competenza;
\item \textbf{Precondizioni:} il richiedente è un Program Manager;
\item \textbf{Postcondizioni:} la lista delle Aree di Competenza è visualizzabile dal Program Manager;
\item \textbf{Estensioni:} il caso d'uso non ha estensioni;
\item \textbf{Inclusioni:} il caso d'uso non ha inclusioni.
\end{itemize}

\subsubsection*{UC5.1 - Vis. singola Area di Competenza}
\begin{figure}[H] 
    \centering 
    \includegraphics[width=0.6\columnwidth]{usecase/UC5.1} 
    \caption{Caso d'Uso 5.1 espanso}
\end{figure}
\begin{itemize}[label=$\circ$]
\item \textbf{Attore:} Program Manager;
\item \textbf{Descrizione:} il Program Manager può visualizzare un'Area di Competenza;
\item \textbf{Precondizioni:} il richiedente è un Program Manager;
\item \textbf{Postcondizioni:} l'Area di Competenza selezionata è visualizzabile dal Program Manager;
\item \textbf{Estensioni:}  UC5.2;
\item \textbf{Inclusioni:} il caso d'uso non ha inclusioni.
\end{itemize}

\subsubsection*{UC5.1.1 - Vis. descrizione Area}
\begin{itemize}[label=$\circ$]
\item \textbf{Attore:} Program Manager;
\item \textbf{Descrizione:} il Program Manager può visualizzare la descrizione di un'Area di Competenza;
\item \textbf{Precondizioni:}  l'Area è visualizzabile dal Program Manager;
\item \textbf{Postcondizioni:} il Program Manager può visualizzare la descrizione dell'Area di Competenza selezionata;
\item \textbf{Estensioni:} non ci sono estensioni;
\item \textbf{Inclusioni:} non ci sono inclusioni.
\end{itemize}

\subsubsection*{UC6 - Vis. lista Clienti}
\begin{figure}[H] 
    \centering 
    \includegraphics[width=0.6\columnwidth]{usecase/UC6} 
    \caption{Caso d'Uso 6 espanso}
\end{figure}
\begin{itemize}[label=$\circ$]
\item \textbf{Attore:} Program Manager;
\item \textbf{Descrizione:} il Program Manager può visualizzare la lista dei Clienti;
\item \textbf{Precondizioni:} il richiedente è un Program Manager;
\item \textbf{Postcondizioni:} la lista dei Clienti è visualizzabile dal Program Manager;
\item \textbf{Estensioni:} il caso d'uso non ha estensioni;
\item \textbf{Inclusioni:} il caso d'uso non ha inclusioni.
\end{itemize}

\subsubsection*{UC6.1 - Vis. singolo Cliente}
\begin{figure}[H] 
    \centering 
    \includegraphics[width=0.6\columnwidth]{usecase/UC6.1} 
    \caption{Caso d'Uso 6.1 espanso}
\end{figure}
\begin{itemize}[label=$\circ$]
\item \textbf{Attore:} Program Manager;
\item \textbf{Descrizione:} il Program Manager può visualizzare un Cliente;
\item \textbf{Precondizioni:} il richiedente è un Program Manager;
\item \textbf{Postcondizioni:} il Cliente selezionato è visualizzabile dal Program Manager;
\item \textbf{Estensioni:} UC6.2;
\item \textbf{Inclusioni:} non ci sono inclusioni.
\end{itemize}

\subsubsection*{UC6.1.1 - Vis. descrizione Cliente}
\begin{itemize}[label=$\circ$]
\item \textbf{Attore:} Program Manager;
\item \textbf{Descrizione:} il Program Manager può visualizzare la descrizione di Cliente;
\item \textbf{Precondizioni:}  il Cliente è visualizzabile dal Program Manager;
\item \textbf{Postcondizioni:} il Program Manager può visualizzare la descrizione del Cliente selezionato;
\item \textbf{Estensioni:} il caso d'uso non ha estensioni;
\item \textbf{Inclusioni:} il caso d'uso non ha inclusioni.
\end{itemize}

\subsubsection*{UC6.1.2 - Vis. codice Cliente}
\begin{itemize}[label=$\circ$]
\item \textbf{Attore:} Program Manager;
\item \textbf{Descrizione:} il Program Manager può visualizzare il codice del Cliente;
\item \textbf{Precondizioni:} il Cliente è visualizzabile dal Program Manager;
\item \textbf{Postcondizioni:} il Program Manager può visualizzare il codice del Cliente selezionato;
\item \textbf{Estensioni:} il caso d'uso non ha estensioni;
\item \textbf{Inclusioni:} il caso d'uso non ha inclusioni.
\end{itemize}


\subsection{Scenario Richieste}
\begin{figure}[H] 
    \centering 
    \includegraphics[width=1.15\columnwidth]{usecase/richieste-general} 
    \caption{Casi d'Uso del scenario Richieste}
\end{figure}

\subsubsection*{UC7 - Creazione Richiesta}
\begin{itemize}[label=$\circ$]
\item \textbf{Attore:} Project Manager;
\item \textbf{Descrizione:} il Project Manager può creare una nuova Richiesta;
\item \textbf{Precondizioni:} il richiedente è un Project Manager;
\item \textbf{Postcondizioni:} la Richiesta è stata creata dal Project Manager con successo;
\item \textbf{Estensioni:} UC8;
\item \textbf{Inclusioni:} il caso d'uso non ha inclusioni.
\end{itemize}

\subsubsection*{UC8 - Vis. errore di validazione Richieste}
\begin{itemize}[label=$\circ$]
\item \textbf{Attore:} Project Manager e Program Manager;
\item \textbf{Descrizione:}  questo caso d'uso descrive anche UC12.2. Viene visualizzato un messaggio di errore in caso vengano eseguite funzionalità con dati non validi. Esso rappresenta i seguenti errori comuni all'interno delle Richieste: dati non validi, filtri non valorizzati, entità associate non valide, risultati nulli o non valorizzati;
\item \textbf{Precondizioni:} il Program Manger o il Project Manager stanno effettuando operazioni con dati non validi;
\item \textbf{Postcondizioni:} l'esecuzione della funzionalità è interrotta e viene visualizzato il messaggio di errore;
\item \textbf{Estensioni:} il caso d'uso non ha estensioni;
\item \textbf{Inclusioni:} il caso d'uso non ha inclusioni.
\end{itemize}

\subsubsection*{UC9 - Eliminazione Richiesta}
\begin{itemize}[label=$\circ$]
\item \textbf{Attore:} Project Manager;
\item \textbf{Descrizione:} il Project Manager può eliminare una Richiesta esistente;
\item \textbf{Precondizioni:} il richiedente è un Project Manager;
\item \textbf{Postcondizioni:} la Richiesta è stata eliminata dal Project Manager con successo;
\item \textbf{Estensioni:} UC8;
\item \textbf{Inclusioni:} il caso d'uso non ha inclusioni.
\end{itemize}

\subsubsection*{UC10 - Modifica Richiesta}
\begin{itemize}[label=$\circ$]
\item \textbf{Attore:} Project Manager;
\item \textbf{Descrizione:} il Project Manager può modificare una Richiesta esistente nella sua totalità sovrascrivendola;
\item \textbf{Precondizioni:} il richiedente è un Project Manager;
\item \textbf{Postcondizioni:} la Richiesta selezionata è stata modificata con successo;
\item \textbf{Estensioni:} UC8;
\item \textbf{Inclusioni:} il caso d'uso non ha inclusioni.
\end{itemize}

\subsubsection*{UC11 - Modifica Priorità Richiesta}
\begin{itemize}[label=$\circ$]
\item \textbf{Attore:} Project Manager;
\item \textbf{Descrizione:} il Project Manager può modificare l'attributo Priorità di una Richiesta esistente inserendo un valore tra: Alta, Media o Bassa;
\item \textbf{Precondizioni:} il richiedente è un Project Manager;
\item \textbf{Postcondizioni:} la Richiesta è stata modificata con successo solo nel campo Priorità dal Project Manager;
\item \textbf{Estensioni:} UC8;
\item \textbf{Inclusioni:} il caso d'uso non ha inclusioni.
\end{itemize}

\subsubsection*{UC12 - Vis. lista Richieste}

\begin{figure}[H] 
    \centering 
    \includegraphics[width=0.7\columnwidth]{usecase/UC12} 
    \caption{Caso d'Uso 12 espanso}
\end{figure}

\begin{itemize}[label=$\circ$]
\item \textbf{Attore:} Project Manager e Program Manager;
\item \textbf{Descrizione:} il Project Manager e il Program Manager possono visualizzare una lista di Richieste dopo aver inserito filtri e/o una parola nella ricerca rapida e aver selezionato se i filtri applicati devono essere congiunti o disgiunti;
\item \textbf{Precondizioni:} il richiedente è un Project Manager o un Program Manager;
\item \textbf{Postcondizioni:} la lista delle Richieste è visualizzabile dal Project Manager e dal Program Manager;
\item \textbf{Estensioni:} UC8;
\item \textbf{Inclusioni:} il caso d'uso non ha inclusioni.
\end{itemize}

\subsubsection*{UC12.1 - Vis. singola Richiesta}

\begin{figure}[H] 
    \centering 
    \includegraphics[width=0.7\columnwidth]{usecase/UC12.1} 
    \caption{Caso d'Uso 12.1 espanso}
\end{figure}

\begin{itemize}[label=$\circ$]
\item \textbf{Attore:} Project Manager e Program Manager;
\item \textbf{Descrizione:} il Project Manager e il Program Manager possono visualizzare la Richiesta selezionata;
\item \textbf{Precondizioni:} la lista delle Richieste è visualizzabile;
\item \textbf{Postcondizioni:} la Richiesta selezionata è visualizzabile dal Program Manager e dal Project Manager;
\item \textbf{Estensioni:} UC12.2;
\item \textbf{Inclusioni:} il caso d'uso non ha inclusioni.
\end{itemize}

\subsubsection*{UC12.1.1 - Vis. dettagli Richiesta}

\begin{itemize}[label=$\circ$]
\item \textbf{Attore:} Project Manager e Program Manager;
\item \textbf{Descrizione:} il Project Manager e il Program Manager possono visualizzare la Richiesta selezionata;
\item \textbf{Precondizioni:} la Richiesta singola è visualizzabile;
\item \textbf{Postcondizioni:} il Project Manager e il Program Manager possono visualizzare i campi di una Richiesta selezionata;
\item \textbf{Estensioni:} il caso d'uso non ha esclusioni;
\item \textbf{Inclusioni:} il caso d'uso non ha inclusioni.
\end{itemize}

\subsubsection*{UC13 - Vis. audit\textsubscript{g} trail Richiesta}

\begin{figure}[H] 
    \centering 
    \includegraphics[width=0.7\columnwidth]{usecase/UC13} 
    \caption{Caso d'Uso 13 espanso}
\end{figure}

\begin{itemize}[label=$\circ$]
\item \textbf{Attore:} Project Manager e Program Manager;
\item \textbf{Descrizione:} il Project Manager e il Program Manager possono visualizzare l'audit trail di una Richiesta selezionata.;
\item \textbf{Precondizioni:} il richiedente è un Project Manager o un Program Manager;
\item \textbf{Postcondizioni:} la traccia di audit della Richiesta selezionata è visualizzabile dal Project Manager e dal Program Manager;
\item \textbf{Estensioni:} UC8;
\item \textbf{Inclusioni:} il caso d'uso non ha inclusioni.
\end{itemize}

\subsubsection*{UC13.1 - Vis. audit singola Richiesta}

\begin{figure}[H] 
    \centering 
    \includegraphics[width=0.8\columnwidth]{usecase/UC13.1} 
    \caption{Caso d'Uso 13.1 espanso}
\end{figure}

\begin{itemize}[label=$\circ$]
\item \textbf{Attore:} Project Manager e Program Manager;
\item \textbf{Descrizione:} il Project Manager e il Program Manager possono visualizzare l'audit di una singola Richiesta;
\item \textbf{Precondizioni:} l'audit trail è visualizzabile;
\item \textbf{Postcondizioni:} l'audit di una singola Richiesta è visualizzabile;
\item \textbf{Estensioni:} il caso d'uso non ha estensioni;
\item \textbf{Inclusioni:} il caso d'uso non ha inclusioni.
\end{itemize}

\subsubsection*{UC13.1.1 - Vis. timestamp\textsubscript{g}}
\begin{itemize}[label=$\circ$]
\item \textbf{Attore:} Project Manager e Program Manager;
\item \textbf{Descrizione:} il Project Manager e il Program Manager possono visualizzare il timestamp di una singola audit di una Richiesta;
\item \textbf{Precondizioni:} la singola audit di una Richiesta è visualizzabile;
\item \textbf{Postcondizioni:} il timestamp della singola audit è visualizzabile;
\item \textbf{Estensioni:} il caso d'uso non ha estensioni;
\item \textbf{Inclusioni:} il caso d'uso non ha inclusioni.
\end{itemize}

\subsubsection*{UC13.1.2 - Vis. operazione}
\begin{itemize}[label=$\circ$]
\item \textbf{Attore:} Project Manager e Program Manager;
\item \textbf{Descrizione:} il Project Manager e il Program Manager possono visualizzare l'operazione di una singola audit di una Richiesta;
\item \textbf{Precondizioni:} la singola audit di una Richiesta è visualizzabile;
\item \textbf{Postcondizioni:} l'operazione della singola audit è visualizzabile;
\item \textbf{Estensioni:} il caso d'uso non ha estensioni;
\item \textbf{Inclusioni:} il caso d'uso non ha inclusioni.
\end{itemize}

\subsubsection*{UC13.1.3 - Vis. Richiesta originale}
\begin{itemize}[label=$\circ$]
\item \textbf{Attore:} Project Manager e Program Manager;
\item \textbf{Descrizione:} il Project Manager e il Program Manager possono visualizzare la singola audit di una Richiesta prima che l'operazione venga eseguita;
\item \textbf{Precondizioni:} la singola audit di una Richiesta è visualizzabile;
\item \textbf{Postcondizioni:} la Richiesta originale della singola audit è visualizzabile;
\item \textbf{Estensioni:} il caso d'uso non ha estensioni;
\item \textbf{Inclusioni:} il caso d'uso non ha inclusioni.
\end{itemize}

\subsubsection*{UC13.1.4 - Vis. Richiesta modificata}
\begin{itemize}[label=$\circ$]
\item \textbf{Attore:} Project Manager e Program Manager;
\item \textbf{Descrizione:} il Project Manager e il Program Manager possono visualizzare la singola audit di una Richiesta dopo che l'operazione è stata eseguita;
\item \textbf{Precondizioni:} la singola audit di una Richiesta è visualizzabile;
\item \textbf{Postcondizioni:} la Richiesta modificata della singola audit è visualizzabile;
\item \textbf{Estensioni:} il caso d'uso non ha estensioni;
\item \textbf{Inclusioni:} il caso d'uso non ha inclusioni.
\end{itemize}

\subsubsection*{UC14 - Esportazione file Excel Richieste in base a filtri}
\begin{itemize}[label=$\circ$]
\item \textbf{Attore:} Project Manager e Program Manager;
\item \textbf{Descrizione:} il Project Manager e il Program Manager possono esportare in un file Excel scaricabile un report delle Richieste in base a dei filtri inseriti;
\item \textbf{Precondizioni:} il richiedente è un Project Manager o un Program Manager;
\item \textbf{Postcondizioni:} il report Excel è stato generato correttamente ed è possibile scaricarlo;
\item \textbf{Estensioni:} UC8;
\item \textbf{Inclusioni:} il caso d'uso non ha inclusioni.
\end{itemize}

\subsubsection*{UC15 - Modifica Stato Richiesta}
\begin{itemize}[label=$\circ$]
\item \textbf{Attore:} Program Manager;
\item \textbf{Descrizione:} il Program Manager può cambiare lo stato di una Richiesta esistente impostandolo in uno dei seguenti valori: Evasa, Non Evasa, Aperta, In corso, Chiusa;
\item \textbf{Precondizioni:} il richiedente è un Program Manager;
\item \textbf{Postcondizioni:} solo lo stato della Richiesta è stato modificato con successo;
\item \textbf{Estensioni:} UC8;
\item \textbf{Inclusioni:} il caso d'uso non ha inclusioni.
\end{itemize}



\subsection{Scenario Pianificazioni}
\begin{figure}[H] 
    \includegraphics[width=1.00\linewidth]{usecase/pianificazioni-general-2} 
    \caption{Casi d'Uso del scenario Pianificazioni}
\end{figure}

\subsection{Scenario Milestone}
\begin{figure}[H] 
    \centering 
    \includegraphics[width=1.05\columnwidth]{usecase/milestone-general} 
    \caption{Casi d'Uso del scenario Milestone}
\end{figure}

\subsubsection*{UC27 - Creazione Milestone}
\begin{itemize}[label=$\circ$]
\item \textbf{Attore:} Program Manager;
\item \textbf{Descrizione:} il Program Manager può creare una nuova Milestone da associare ad una Pianificazione;
\item \textbf{Precondizioni:} il richiedente è un Program Manager;
\item \textbf{Postcondizioni:} la Milestone è stata creata dal Program Manager con successo;
\item \textbf{Estensioni:} UC28;
\item \textbf{Inclusioni:} il caso d'uso non ha inclusioni.
\end{itemize}

\subsubsection*{UC28 - Vis. errore di validazione Milestone}
\begin{itemize}[label=$\circ$]
\item \textbf{Attore:} Program Manager;
\item \textbf{Descrizione:} questo caso d'uso descrive anche UC30.2. Viene visualizzato un messaggio di errore in caso vengano eseguite funzionalità con dati non validi. Esso rappresenta i seguenti errori comuni all'interno delle Milestone: dati non validi, filtri non valorizzati, entità associate non valide, risultati nulli o non valorizzati;
\item \textbf{Precondizioni:} il Program Manager sta effettuando operazioni con dati non validi;
\item \textbf{Postcondizioni:} l'esecuzione della funzionalità è interrotta e viene visualizzato il messaggio di errore;
\item \textbf{Estensioni:} il caso d'uso non ha estensioni;
\item \textbf{Inclusioni:} il caso d'uso non ha inclusioni.
\end{itemize}

\subsubsection*{UC29 - Modifica Tipo Milestone}
\begin{itemize}[label=$\circ$]
\item \textbf{Attore:} Program Manager;
\item \textbf{Descrizione:}  il Program Manager può modificare il Tipo di una Milestone in due possibili valori: Alert o Reminder;
\item \textbf{Precondizioni:} il richiedente è un Program Manager;
\item \textbf{Postcondizioni:} la Milestone è stata modificata con successo solo nel campo
Tipo dal Program Manager;
\item \textbf{Estensioni:} UC28;
\item \textbf{Inclusioni:} il caso d'uso non ha inclusioni.
\end{itemize}

\subsubsection*{UC30 - Vis. lista Milestone}
\begin{figure}[H] 
    \centering 
    \includegraphics[width=0.70\columnwidth]{usecase/UC30} 
    \caption{Casi d'Uso 30 espanso}
\end{figure}
\begin{itemize}[label=$\circ$]
\item \textbf{Attore:} Program Manager;
\item \textbf{Descrizione:} il Program Manager può visualizzare una lista di Milestone dopo aver inserito filtri e/o una parola nella ricerca rapida e aver selezionato se i filtri applicati devono essere congiunti o disgiunti;
\item \textbf{Precondizioni:} il richiedente è un Program Manager;
\item \textbf{Postcondizioni:} la lista delle Milestone è visualizzabile dal Program Manager;
\item \textbf{Estensioni:} UC28;
\item \textbf{Inclusioni:} il caso d'uso non ha inclusioni.
\end{itemize}

\subsubsection*{UC30.1 - Vis. singola Milestone}
\begin{figure}[H] 
    \centering 
    \includegraphics[width=0.70\columnwidth]{usecase/UC30.1} 
    \caption{Casi d'Uso 30.1 espanso}
\end{figure}
\begin{itemize}[label=$\circ$]
\item \textbf{Attore:} Program Manager;
\item \textbf{Descrizione:} il Program Manager può visualizzare la Milestone selezionata;
\item \textbf{Precondizioni:} la lista delle Milestone è visualizzabile;
\item \textbf{Postcondizioni:} la Milestone selezionata è visualizzabile dal Program Manager;
\item \textbf{Estensioni:} UC30.2;
\item \textbf{Inclusioni:} il caso d'uso non ha inclusioni.
\end{itemize}

\subsubsection*{UC30.1.1 - Vis. dettagli Milestone}
\begin{itemize}[label=$\circ$]
\item \textbf{Attore:} Program Manager;
\item \textbf{Descrizione:} il Program Manager può visualizzare la Milestone selezionata;
\item \textbf{Precondizioni:} la Milestone singola è visualizzabile;
\item \textbf{Postcondizioni:} il Project Manager può visualizzare i campi di una Milestone selezionata;
\item \textbf{Estensioni:} il caso d'uso non ha esclusioni;
\item \textbf{Inclusioni:} il caso d'uso non ha inclusioni.
\end{itemize}

\subsubsection*{UC31 - Eliminazione Milestone}
\begin{itemize}[label=$\circ$]
\item \textbf{Attore:} Program Manager;
\item \textbf{Descrizione:} il Program Manager può eliminare una Milestone esistente;
\item \textbf{Precondizioni:} il richiedente è un Program Manager;
\item \textbf{Postcondizioni:} la Milestone è stata eliminata dal Program Manager con successo;
\item \textbf{Estensioni:} UC28;
\item \textbf{Inclusioni:} il caso d'uso non ha inclusioni.
\end{itemize}

\section{Tracciamento dei requisiti}
Di seguito elenco i requisiti funzionali estrapolati dallo studio degli use case.\\
Per poterli elencare e distinguere è stata utilizzata il seguente codice identificativo:
\begin{center}
\textbf{RF[OB/DE/OP]-[Numero progressivo]}
\end{center}
All'interno del codice possiamo osservare RF che significa \textit{Requisito funzionale}, mentre le altre abbreviazioni indicano l'importanza del requisito:
\begin{itemize}
\item \textit{OB}, obbligatorio;
\item \textit{DE}, desiderabile;
\item \textit{OP}, opzionale.
\end{itemize}
La tabella elenca i requisiti per Codice, come descritto, Descrizione del requisito, e la Fonte che può essere una decisione interna o un use case.
\\\\
\setlength{\arrayrulewidth}{0.3mm}
\renewcommand{\arraystretch}{2.5}
\begin{center}
\rowcolors{1}{white}{mygray}
\begin{longtable}{p{2.0cm}|p{8cm}|p{2.7cm}}
\textbf{Codice}  & \textbf{Descrizione} & \textbf{Fonte}\\
\hline
\hypertarget{rf01}{RFOP-01}  &	 Deve essere predisposto un accesso controllato al sistema, che consenta solo a utente autorizzati di effettuare determinate operazioni & Decisione interna \\ 
RFOB-02  & Il sistema permette la visualizzazione dei dettagli di una Risorsa selezionata & UC1 \\ 
RFOB-03  & Il sistema permette la visualizzazione del nome della Risorsa & UC1.1 \\ 
RFOB-04  & Il sistema permette la visualizzazione del cognome della Risorsa & UC1.2 \\ 
RFOB-05  & Il sistema permette la visualizzazione dell'email della Risorsa & UC1.3 \\ 
RFOB-06  & Il sistema controlla se si sta eseguendo operazioni nell'ambito Anagrafiche con dati non validi & UC2 \\ 
RFOB-07  & Il sistema controlla se il dato fornito non ha portato ad alcun risultato & UC2 \\ 
RFOB-08  & Il sistema permette la visualizzazione della lista di Skills & UC3 \\ 
RFOB-09  & Il sistema permette la visualizzazione di una singola Skill dalla lista & UC3.1 \\ 
RFOB-10  & Il sistema permette la visualizzazione diretta di una singola Skill & UC3.1 \\ 
RFOB-11  & Il sistema verifica che la richiesta alla singola Skill sia valida & UC3.2 \\ 
RFOB-12  & Il sistema permette la visualizzazione della descrizione della Skill & UC3.1.1 \\ 
RFOB-13  & Il sistema permette la visualizzazione della lista di Ruoli & UC4 \\ 
RFOB-14  & Il sistema permette la visualizzazione di uno singolo Ruolo dalla lista & UC4.1 \\ 
RFOB-15  & Il sistema permette la visualizzazione diretta di un singolo Ruolo & UC4.1 \\ 
RFOB-16  & Il sistema verifica che la richiesta al singolo Ruolo sia valida & UC4.2 \\ 
RFOB-17  & Il sistema permette la visualizzazione della descrizione del Ruolo & UC4.1.1 \\ 
RFOB-18  & Il sistema permette la visualizzazione della lista di Area di Competenza & UC5 \\ 
RFOB-19  & Il sistema permette la visualizzazione di una singola Area di Competenza dalla lista & UC5.1 \\ 
RFOB-20  & Il sistema permette la visualizzazione diretta di una singola Area di Competenza & UC5.1 \\ 
RFOB-21  & Il sistema verifica che la richiesta alla singola Area di Competenza sia valida & UC5.2 \\ 
RFOB-22  & Il sistema permette la visualizzazione della descrizione dell'Area di Competenza & UC5.1.1 \\ 
RFOB-23  & Il sistema permette la visualizzazione della lista di Clienti & UC6 \\ 
RFOB-24  & Il sistema permette la visualizzazione di uno singolo Cliente dalla lista & UC6.1 \\ 
RFOB-25  & Il sistema permette la visualizzazione diretta di un singolo Cliente & UC6.1 \\ 
RFOB-26  & Il sistema verifica che la richiesta al singolo Cliente sia valida & UC6.2 \\ 
RFOB-27  & Il sistema permette la visualizzazione della descrizione del Cliente & UC6.1.1\\ 
RFOB-28  & Il sistema permette la visualizzazione del codice del Cliente & UC6.1.2 \\ 

RFOB-29  & Il sistema permette la creazione di una nuova Richiesta di figure professionali & UC7 \\ 
RFOB-30  & Il sistema controlla che il Richiedente esista & UC8 \\ 
RFOB-31  & Il sistema controlla che le Figure richieste e le Skill richieste siano coerenti & UC8 \\ 
RFOB-32  & Il sistema controlla che il Cliente associato sia corretto & UC8 \\ 
RFOB-33  & Il sistema controlla che il Progetto associato sia corretto & UC8 \\ 
RFOB-34  & Il sistema controlla che una Richiesta che si vuole eliminare non sia associata ad una Pianificazione & UC8 \\ 
RFOB-35 & Il sistema controlla che i nuovi valori per la modifica di un singolo campo siano corretti & UC8,UC17,UC28\\
RFOB-36  & Il sistema controlla che i filtri inseriti siano valorizzati & UC8,UC17,UC28 \\ 
RFOB-37  & Il sistema controlla che l'entità su cui si va ad operare esista & UC8,UC17,UC28 \\ 


RFOB-38  & Il sistema permette l'eliminazione di una Richiesta & UC9 \\ 
RFOB-39  & Il sistema permette la modifica totale di una Richiesta esistente sovrascrivendola & UC10 \\ 
RFOB-40  & Il sistema permette la modifica del solo campo Priorità di una Richiesta esistente & UC11 \\ 
RFOB-41  & Il sistema permette la visualizzazione della lista di Richieste risultanti in seguito ad una richiesta formata da: filtri forniti dall'utente, parola da inserire nella ricerca rapida e se i filtri devono essere congiunti o disgiunti & UC12 \\ 
RFOB-42  & Il sistema permette la visualizzazione di una singola Richiesta dalla lista & UC12.1 \\ 
RFOB-43  & Il sistema permette la visualizzazione diretta di una singola Richiesta & UC12.1 \\ 
RFOB-44  & Il sistema verifica che la richiesta alla singola Richiesta sia valida & UC12.2 \\ 
RFOB-45  & Il sistema permette la visualizzazione dei dettagli della Richiesta selezionata & UC12.1.1 \\ 



\hypertarget{rf46}{RFDE-46}  & Il sistema permette la visualizzazione della traccia di audit di una Richiesta & UC13 \\ 
RFDE-47  & Il sistema permette la visualizzazione di un audit di una Richiesta della traccia & UC13.1 \\ 
RFDE-48  & Il sistema permette la visualizzazione del timestamp dell'operazione in un singolo audit di un audit trail & UC13.1.1 \\ 
RF4DE-9  & Il sistema permette la visualizzazione dell'operazione in un singolo audit di un audit trail & UC13.1.2 \\ 
RFDE-50  & Il sistema permette la visualizzazione della Richiesta originale in un singolo audit di un audit trail & UC13.1.3 \\ 
RFDE-51  & Il sistema permette la visualizzazione della Richiesta modificata in un singolo audit di un audit trail & UC13.1.4 \\ 



\hypertarget{rf52}{RFDE-52}  & Il sistema permette di esportare un file Excel contenente le Richieste risultanti dai filtri inseriti dall'utente nella richiesta & UC14 \\ 
RFOB-53  & Il sistema permette la modifica dell'attributo Stato di una Richiesta & UC15 \\ 

RFOB-54  & Il sistema permette la creazione di una nuova Pianificazione & UC16 \\  
RFOB-55  & Il sistema controlla che il Progetto associato sia corretto & UC17 \\ 
RFOB-56  & Il sistema controlla che la Risorsa non sia occupata in quel Ruolo & UC17 \\ 
RFOB-57  & Il sistema controlla che la Risorsa possa svolgere il Ruolo richiesto & UC17 \\ 
RFOB-58  & Il sistema permette l'eliminazione di una Pianificazione  & UC18 \\ 
RFOB-59 & Il sistema permette la modifica totale di una Pianificazione sovrascrivendola & UC19\\
RFOB-60 & Il sistema permette la modifica delle date di una Pianificazione & UC20\\
RFOB-61 & Il sistema permette la modifica del campo Festivi di una Pianificazione & UC21\\
RFDE-62 & Il sistema permette di esportare un file Excel contenente le Pianificazioni risultanti dai filtri inseriti dall'utente nella richiesta & UC22\\
\hypertarget{rf63}{RFOP-63} & Il sistema permette di esportare un file Excel contenente lo storico delle Pianificazioni relative ad un Progetto, mostrando le Risorse allocate e l'effettivo impiego di queste nelle attività & UC23\\
\hypertarget{rf64}{RFOP-64} &  Il sistema permette di esportare un file Excel contenente lo storico delle Pianificazioni di una singola Risorsa, visualizzando data fine e data inizio e le relative mansioni & UC24\\

\hypertarget{rf65}{RFDE-65} & Il sistema permette la visualizzazione della traccia di audit di una Pianificazione & UC25\\
RFDE-66 & Il sistema permette la visualizzazione di un audit di una Pianificazione della traccia & UC25.1\\
RFDE-67 & Il sistema permette la visualizzazione del timestamp dell'operazione in un singolo audit di un audit trail & UC25.1.1\\
RFDE-68 & Il sistema permette la visualizzazione dell'operazione in un singolo audit di un audit trail & UC25.1.2\\
RFDE-69 & Il sistema permette al visualizzazione della Pianificazione originale in un singolo audit di un audit trail & UC25.1.3\\
RFDE-70 & Il sistema permette al visualizzazione della Pianificazione modificata in un singolo audit di un audit trail & UC25.1.4\\
RFOB-71 & Il sistema permette la visualizzione della lista di Pianificazioni risultanti in seguito ad una richiesta formata da: filtri forniti dall'utente, parola da inserire nella ricerca rapida e se i filtri devono essere congiunti o disgiunti & UC26\\
RFOB-72 & Il sistema permette la visualizzazione di una singola Pianificazione dalla lista & UC26.1\\
RFOB-73 & Il sistema permette la visualizzazione diretta di una singola Pianificazione & UC26.1\\
RFOB-74 & Il sistema verifica che la richiesta alla singola Pianificazione sia valida & UC26.2\\
RFOB-75 & Il sistema permette la visualizzazione dei dettagli della Pianificazione selezionata & UC26.1.1\\
RFOB-76 & Il sistema permette di creare una nuova Milestone da associare ad una Pianificazione & UC27\\
RFOB-77 & Il sistema controlla che il Progetto, il Commerciale associati alla Milestone siano corretti & UC28\\
RFOB-78 & Il sistema controlla che la Milestone che si vuole eliminare non sia associata ad una Pianificazione & UC28\\
RFOB-79 & Il sistema permette la modifica totale di una Milestone sovrascrivendola & UC29\\
RFOB-80 & Il sistema permette la visualizzazione della lista di Milestone risultanti in seguito ad una richiesta formata da: filtri forniti dall'utente, parola da inserire nella ricerca rapida e se i filtri devono essere congiunti o disgiunti & UC30\\
RFOB-81 & Il sistema permette la visualizzazione di una singola Milestone dalla lista & UC30.1\\
RFOB-82 & Il sistema permette la visualizzazione diretta di una singola Milestone & UC30.1\\
RFOB-83 & Il sistema verifica che la richiesta alla singola Pianificazione sia valida & UC30.2\\
RFOB-84 & Il sistema permette la visualizzazione dei dettagli della Pianificazione selezionata & UC30.1.1\\
RFOB-85 & Il sistema permette l'eliminazione di una Milestone & UC31\\
\hline
\hiderowcolors
\caption{Lista dei Requisiti}
\label{tab:Requisiti}
\end{longtable}
\end{center}



 

