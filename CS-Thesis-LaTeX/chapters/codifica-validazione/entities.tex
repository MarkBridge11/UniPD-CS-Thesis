\subsection{Entities}

\texttt{FOTO DEI PACKAGE ENTITA}\\

\noindent All'interno del package Entities troviamo le entità.\\
Per entità si intendono tutte quelle classi Java che definiscono i modelli di dati dell'applicazione. Queste classi vengono annotate con le seguenti annotazioni JPA utili a stabilire come la classe venga associata a una tabella presente nel database relazionale. Ogni istanza di un'entità rappresenta una riga nella tabella relativa.
\begin{itemize}
\item \textit{@Entity}, per mappare le classi Java che rappresentano una tabella in un database si inserisce questa notazione specificando a class level\textsubscript{g}.
\item \textit{@Table}, nella maggior parte dei casi il nome di una tabella nel database e il nome dell'entità non sono gli stessi. Per questo motivo è stata utilizzata la seguente annotazione per specificare il nome della tabella.
\end{itemize} 

\noindent \texttt{FOTO ENTITA ELABORATA}\\

\noindent Spiegazione annotazioni.\\

\noindent \texttt{FOTO DI UNA ENTITA DTO}\\

\noindent Come funziona e a cosa serve.\\