\chapter{Codifica e validazione}
\label{cap:codifica-validazione}

\section{Scopo del capitolo}
Nel seguente capitolo si può osservare il lavoro dei giorni seguenti alla Progettazione fino alla fine del periodo di tirocinio. In questo capitolo viene illustrato lo sviluppo dell'API REST, come è stato organizzato il codice parlando di classi e package, la codifica dei moduli più rilevanti ed infine la validazione. \\

\section{Endpoint sviluppati dell'API REST}
In questa sezione si trovano le descrizioni di tutti gli endpoint implementati, suddivisi in base all'ambito di interesse del servizio. Sarà fornito anche il verbo HTTP e il percorso necessario per effettuare ciascuna richiesta.\\
I verbi standard forniti da HTTP utilizzati applicati all'API REST sono i seguenti:
 
\setlength{\arrayrulewidth}{0.3mm}
\renewcommand{\arraystretch}{2.5}
\begin{center}
\begin{longtable}{p{2cm}|p{8cm}}
\textbf{Verbo}  & \textbf{Descrizione} \\ \midrule
\rowcolor{mygray} 
POST   & Utilizzato per inviare dati al server al fine di creare una nuova risorsa ed aggiungerla all'insieme corrente\\
GET    & Utilizzato per richiedere dati al server in merito ad una o più risorse senza modificarle          \\
\rowcolor{mygray}
DELETE &   Utilizzato per eliminare una risorsa specifica dal server          \\
PUT    &   Utilizzato per aggiornare totalmente una risorsa e viene utilizzato quando si vuole sostituire completamente una risorsa          \\
\rowcolor{mygray}
PATCH  &     Utilizzato per effettuare aggiornamenti parziali a una risorsa esistente        \\ \bottomrule
\caption{Verbi Standard HTTP utilizzati}
\label{tab:verbi-http}
\end{longtable}
\end{center}

%Per garantire una lettura chiara dello stato dell'operazione richiesta, sono stati utilizzati i seguenti codici di risposta HTTP. Questi codici 

\noindent Alla fine del progetto questi sono gli endpoint che sono stati sviluppati:\\\\
\texttt{GLI ENDPOINT IN UNA TABELLA CON VERBO + PATH + DESCRIZIONE}\\

\section{Organizzazione del codice}
\subsection{Package common e config}
I package contengono i seguenti file .java:

\begin{figure}[H] 
    \centering 
    \includegraphics[width=0.6\columnwidth]{common-config} 
    \caption{Package common e config}
\end{figure}


\section{Packages Common e config}
I package contengono i seguenti file .java:

\begin{figure}[H] 
    \centering 
    \includegraphics[width=0.4\columnwidth]{common-config} 
    \caption{Package common e config}
\end{figure}
\subsection{Common}
\begin{itemize}
\item \texttt{OmiPlanException}, eccezione personalizzata utilizzata per gestire le \textit{RuntimeException}. Classe formata da un HttpStatus per mostrare lo stato di risposta HTTP e dal messaggio fornito al lancio dell'eccezione;
\item \texttt{PaginationDirection}, classe enumerativa per gestire la direzione della paginazione\textsubscript{g} (ASC o DESC);
\item \texttt{SortDirectionVerifierConverter}, classe che implementa l'interfaccia \textit{Converter<S,T>} (componente Java utilizzato quando si lavora con strutture dati o oggetti che devono essere trasformati o adattatati in tipi diversi) per verificare che la direzione inserita sia ASC o DESC e gestire la \textit{RunTimeException} in caso non sia una variabile enum;
\item \texttt{StringToEnumConverter}, classe che implementa l'interfaccia \textit{Converter<S,T>} per convertire la direzione Stringa inserita nell'enum della direzione di pagina.
\end{itemize}

\subsection{Config}
\begin{itemize}
\item \texttt{OmiPlanExceptionHandler}, handler con il compito di gestire le eccezioni runtime, segnando quelle non gestite come "Unexpected error". Questa classe è annotata con \textit{@ControllerAdvice}, poiché definisce una classe che gestisce in maniera centralizzata le eccezioni. Contiene due metodi annotati con \textit{@ExceptionHandler} che gestiscono rispettivamente le \textit{RunTimeException} e le altre eccezioni \textit{Exception};
\item \texttt{WebConfig}, classe annotata con l'annotazione \textit{@Configuration} indicando che è una classe di configurazione e che contiene definizioni di bean o altre configurazioni necessarie per l'applicazione. Essa infatti estende \textit{WebMvcConfigurer} che consente di modificare le configurazioni predefinite di Spring MVC, in questo caso aggiungendo i converter sopra citati.
\end{itemize}

\section{Package Entities}

\begin{figure}[H] 
    \centering 
    \includegraphics[width=0.4\columnwidth]{entities-package2} 
    \caption{Package entità}
\end{figure}

\noindent All'interno del package Entities troviamo le entità. Per entità si intendono tutte quelle classi Java che definiscono i modelli di dati dell'applicazione.\\
Queste classi vengono annotate con annotazioni JPA utili a stabilire come la classe venga associata a una tabella presente nel database relazionale. Ogni istanza di un'entità rappresenta una riga nella tabella relativa.\\
Di tutte le tabelle da me create è stata mappata ogni relazione tra tabelle e campo, mentre per ogni tabella del database aziendale sono stati mappati tutti i campi e le relazioni ad altre tabelle che potevano tornarmi utili.\\
In ogni entità troviamo setters e getters, costruttori con e senza argomenti.\\

\subsection{Esempio di un'entità}
\begin{figure}[H] 
    \centering 
    \includegraphics[width=0.9\columnwidth]{esempio-entita} 
    \caption{Esempio di mappatura di un'entità}
\end{figure}

\noindent Nell'immagine qui sopra possiamo notare uno snippet dell'entità Pianificazione in cui sono state utilizzate le seguenti annotazioni:
\begin{itemize}
\item \textit{@Entity}, per mappare le classi Java che rappresentano una tabella in un database si inserisce questa notazione specificando a class level\textsubscript{g}.
\item \textit{@Table}, nella maggior parte dei casi il nome di una tabella nel database e il nome dell'entità non sono gli stessi. Per questo motivo è stata utilizzata la seguente annotazione per specificare il nome della tabella;
\item \textit{@Column}, utilizzata per mappare una campo di una classe a una colonna di una tabella nel database. Questa annotazione possiede parametri come il "nome" per specificare il nome della colonna a cui è associato il campo;
\item \textit{@Id}, per identificare un campo all'interno di una classe come chiave primaria in una tabella del database viene utilizzata questa annotazione;
\item \textit{@Temporal}, utilizzata per specificare se un campo di tipo Date dovrebbe essere mappato come \textit{TemporalType.DATE}, per una data senza orario, \textit{TemporalType.TIME} per un orario senza data e infine \textit{TemporalType.TIMESTAMP}, utilizzato nella tabella di log di Pianificazione per mappare un campo con data e orario.
\end{itemize}
\subsection{Mappatura delle relazioni}
\begin{figure}[H] 
    \centering 
    \includegraphics[width=0.9\columnwidth]{esempio-relazioni} 
    \caption{Esempio di mappatura delle relazioni della tabella RichiestaTestata}
\end{figure}
\noindent Per mantenere le relazioni tra le tabelle sono state utilizzate le annotazioni come nello snippet qui sopra raffigurante una porzione della classe Java che mappa la tabella RichiestaTestata.\\
In casi in cui entrambi i lati della relazione necessitavano per motivi implementativi, come la visualizzazione da entrambe le parti dell'informazione, di mappare la relazione opposta, veniva dichiarata una relazione bidirezionale, in cui entrambi i lati mappavano la relazione e diventava importante gestire correttamente la sincronizzazione tra le due entità.
\begin{itemize}
\item \textbf{uno-a-molti}, relazione in cui il campo associato sarà una lista di oggetti della classe opposta (in relazione all'esempio, una RichiestaTestata è associata a più RichiesteDettaglio), mentre nella relazione molti-a-uno sarà un oggetto singolo della classe opposta annotato con \textit{@JoinColumn} con \texttt{name} uguale a quello del campo nella tabella del database (in relazione all'esempio, una RichiestaTestata ha un solo Richiedente).\\
Nell'esempio è stato utilizzato il parametro \texttt{mappedBy} per mappare la relazione opposta e \texttt{orphanRemoval} per poter implementare la cancellazione di una RichiestaTestata garantendo l'eliminazione di tutte le RichiesteDettaglio associate al momento della rimozione.
\item \textbf{uno-a-uno}, relazione in cui un record della tabella è associato ad un solo record dell'altra tabella. Anche in questo caso se si vuole mappare da entrambi le parti la relazione bisogna utilizzare lo stesso principio dichiarato nella relazione \textit{@OneToMany}, solo che da entrambi le parti avranno come campo un oggetto della classe opposta.\\
Non sono state identificate relazioni uno-a-uno nel database.
\item \textbf{molti-a-molti}, relazione in cui viene mappata la tabella di join presente nel database tra le due tabelle, utilizzando il codice che si vede in esempio.
Per gestire operazioni di rimozione tra le due tabelle è stato utilizzato un metodo annotato con \textit{@PreRemove} nella classe TipoSkill, che entra in azione quando una RichiestaTestata viene eliminata, assicurando che la disconnessione avvenga in modo appropriato.
%\item \textbf{relazione bidirezionale}, relazione in cui è importante gestire correttamente la sincronizzazione tra le due direzioni della relazione, dato che quando si aggiorna una parte dell'entità bisogna aggiornare anche l'altra. Per mappare una relazione non è necessario mappare entrambi i lati della relazione, ma ritorna utile solo in base quello che si va ad implementare.
\end{itemize}
\subsection{Subpackage dtoentities}
All'interno del package "dtoentities" troviamo tutte quelle classi DTO delle entità di cui servivano soltanto determinate informazioni da restituire all'utente, evitando così ridondanza o informazioni superflue. Queste classi non contengono annotazioni, ma sono semplicemente degli oggetti contenenti i campi semplificati delle entità. A loro volta possono contenere altri DTO di altre entità in base alle relazioni che hanno.
\begin{figure}[H] 
    \centering 
    \includegraphics[width=0.9\columnwidth]{merge-dtopianificazione} 
    \caption{Esempio di classe DTO di Pianificazione}
\end{figure}
\noindent Ogni entità DTO possiede tre costruttori: costruttore senza argomenti e con argomenti e infine un costruttore che velocizzava la conversione da entità ad entità DTO (come mostrato nell'immagine qui sopra).

\subsection{Package dto}
Per la composizione delle richieste che il client invia o delle risposte che il server manda, sono state create vari tipi di requests e responses.\\
Ogni oggetto di risposta è composto da un oggetto data e un oggetto metadata, mentre gli oggetti di richiesta possono variare. Se la richiesta è fatta per un endpoint che restituisce una lista di risultati, possiamo avere tre casistiche:
\begin{itemize}
\item body\textsubscript{g} formato da un oggetto data, contenente informazioni utili alla risposta, e metadata che contiene i dati di paginazione\textsubscript{g};
\item nessun body, ma soltanto i dati di paginazione passati come parametri di query\textsubscript{g};
\item path variable\textsubscript{g} che solitamente fa riferimento ad un ID, nei parametri di query e dati di paginazione.
\end{itemize}
Se invece la richiesta è costruita per un endpoint che restituisce un singolo oggetto nell'oggetto data, non viene utilizzato alcun body o parametri di query, ma solo una path variable.\\
Questa suddivisione data e metadata è un modello comune nella progettazione API per separare i dati effettivi o dati che forniscono informazioni per ottenere un tipo di risposta, dalle informazioni aggiuntive che descrivono il risultato o aggiungono caratteristiche alla richiesta.\\
Questo package è formato da vari subpackage\textsubscript{g}.
\subsubsection{Common}
\begin{figure}[H] 
    \centering 
    \includegraphics[width=1.0\columnwidth]{dto-sub-common} 
    \caption{Subpackage common}
\end{figure}
Questo subpackage contiene la response di base, response paginata, la response utilizzata nell'eccezione personalizzata, una request per la paginazione e i vari metadata utilizzati nelle responses. Ogni metadata contiene un boolean che indica se si è andati in errore (true) o meno (false), il messaggio di errore e una lista di \texttt{validation}, contenente informazioni di validazione dei dati.\\
È stata creata una classe \texttt{Pagination} per avere un controllo personalizzato sulla Paginazione.
\subsubsection{Requests e Body}
\begin{figure}[H] 
    \centering 
    \includegraphics[width=1.0\columnwidth]{dto-sub-requests-body-2} 
    \caption{Subpackage requests e body}
\end{figure}
Questi due subpackages sono correlati tra di loro dato che ogni body ha come attributo una request presa dal subpackage requests.
Esistono due tipi di body:
\begin{itemize}
\item ogni \texttt{Request} singola contiene una request DTO formata dai campi utili che l'utente deve inserire per eseguire un determinato endpoint;
\item ogni \texttt{ListRequest} è formata da un oggetto data di tipo \texttt{ListRequestInternal}, che contiene tre parametri: filters, lista di oggetti Filter formati da campo-valore, una stringa q (quicksearch) utile nella query personalizzata per cercare in determinati campi quello che l'utente inserisce e infine andOperator per impostare i filtri in And o in Or nella query. 
\end{itemize}
All'interno del subpackage requests troviamo anche \texttt{PatchRequest}, contenente un unico campo corrispondente ad una lista di \texttt{PatchField}. Questa richiesta viene utilizzata nella richieste PATCH, in cui è possibile inserire uno o più valori in base a cosa si va a modificare.

\subsubsection{Responses}
\begin{figure}[H] 
    \centering 
    \includegraphics[width=0.9\columnwidth]{dto-sub-responses} 
    \caption{Subpackage responses e dtoresponses}
\end{figure}
Il package responses contiene tutte le response per entità. Ogni response eredita la classe astratta \texttt{response} se è una risposta singola, altrimenti eredita la classe \texttt{PaginatedResponse} se è una lista di risposte. Questo permette all'utente di visualizzare la risposta dopo aver interagito con un'endpoint, formata da data e metadata.\\
Ogni response contiene un solo attributo che corrisponde all'oggetto data e può essere una responses DTO del subpackage dtoresponses o un'entità DTO del subpackage delle entità. Se si è di fronte ad una \texttt{ListResponse} l'attributo sarà una lista, altrimenti un oggetto singolo.

\subsection{Repository-Service-Controller}

\begin{figure}[H] 
    \centering 
    \includegraphics[width=0.85\columnwidth]{csp_img} 
    \caption{Schema di collegamento tra i componenti Repository,Service e Controller}
\end{figure}

\noindent Funzionamento.\\

\noindent \texttt{FOTO DI COME SONO NELLA DIRECTORY}\\

\noindent Presentazione di una Repository in particolare con spiegazione annotazioni.\\\\
Presentazione di una Service in particolare con spiegazione annotazioni.\\\\
Presentazione di una Controller (questo più per swagger ui) + ControllerImpl in particolare con spiegazione annotazioni.\\


\section{Validazione}
Nell’ultimo giorno di tirocinio si è tenuto un incontro conclusivo insieme al tutor Antonio Fasolato e al \textit{RUOLO\_LAVORATIVO} Giovanni Incammicia, a conferma definitiva della conformità alle specifiche richieste del prodotto finale. Durante questa riunione, è stata effettuata una revisione dell'Analisi dei Requisiti fornita all'inizio del progetto, con particolare attenzione ai requisiti di mia competenza, al fine di verificare il loro soddisfacimento. Tuttavia, per scrivere il codice dell’API, è stato utilizzato principalmente un disegno iniziale dell’API e l’Analisi dei Requisiti fornita è stata utilizzata per aiutare la comprensione di alcuni endpoint. Questo approccio ha comportato alcune discrepanze tra le funzionalità implementate e i requisiti specificati. Rispetto all’Analisi dei Requisiti mancavano degli attributi che il Front-end necessitava o alcune funzionalità che necessitavano implementazioni particolari o di collegamenti a servizi esterni di cui non c’era tempo a sufficienza per poter studiare. Nonostante queste mancanze, analizzando le funzionalità implementate e le response fornite, è stato ritenuto sufficiente per il soddisfacimento dei requisiti di cui gli endpoint sviluppati facevano parte.\\
A causa delle problematiche da me sopra menzionate, non è stato possibile condurre una fase di Collaudo effettiva per valutare l'interazione tra il Front-end e il Back-end.


