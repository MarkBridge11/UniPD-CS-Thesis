\chapter{Conclusioni}
\label{cap:conclusioni}

\section{Consuntivo finale}
Da completare.

\section{Principali problematiche e soluzioni}
\subsection{Problematiche}
\begin{enumerate}
\item Per cominciare lo sviluppo del progetto è stata presa come base un’Analisi dei Requisiti del progetto completo (lato Front-end integrato con lato Back-end) insieme ad una bozza dell'API che dovevo andare a sviluppare. È stato dunque difficile riuscire a far concordare quello che l’API suggeriva di implementare con il requisito completo che l’Analisi dei Requisiti forniva ed arrivare al codice da sviluppare;
\item Le informazioni che le funzionalità da me implementate necessitavano veniva recuperate da tabelle da me create. Tuttavia queste tabelle necessitavano di dati da altre tabelle già esistenti nel database aziendale. La problematica era quella di identifcare quali tabelle del database aziendale dovevo collegare alle tabelle di mia creazione; 
\item Nel periodo finale di tirocinio mi sono interfacciato con chi lavorava nel lato Front-End per poter cominciare ad integrare il mio lavoro ed avere una visione più completa di quanto implementato. Per poter fare questo bisognava quindi avere una documentazione dell’API chiara e concordata tra le parti.
\end{enumerate}

\subsection{Soluzoni}
\begin{enumerate}
\item Per lo sviluppo è stato preso in considerazione il mock iniziale dell’API , utilizzando come consulto per un chiarimento sulla funzionalità fornita dall’endpoint, l’Analisi dei Requisiti, come spiegato nella sezione di \hyperlink{validation}{Validazione}. A fine del tirocinio è stata poi creato un nuovo documento dell'API, come descritto nella sezione dello \hyperlink{swagger}{Swagger} nel capitolo di Sviluppo dell’API REST, considerandolo come documento unico per la consultazione dell’API sviluppata;
\item Per avere una comprensione migliore del database aziendale mi è stato fornito un backup del database di testing aziendale con dati fittizi in cui c’era la possibilità di testare come volevo. Grazie al tutor e agli altri collaboratori ho avuto una visione più completa di come funzionasse il database aziendale e a quali tabelle dovevo fare riferimento;
\item Per poter comunicare correttamente con gli altri sviluppatori è stato creato il file Swagger. Tuttavia, come spiegato nel capitolo di \hyperlink{validation}{Validazione}, a fine progetto, anche se l’API era chiara e documentata, ci sono state delle discrepanze tra le funzionalità implementate che non hanno permesso la fase di Collaudo.
\end{enumerate}


\section{Possibili estensioni}
Da completare.

\section{Conoscenze acquisite}
Da completare.

\section{Valutazione personale}
Da completare.
