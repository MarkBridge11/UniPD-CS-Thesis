\hypertarget{tecnologie}{\chapter{Background tecnologico}}
\label{cap:tecnologie}

Questo capitolo tratta delle tecnologie e gli strumenti di sviluppo e di supporto utilizzati per la realizzazione del prodotto.\\ L'apprendimento delle seguenti tecnologie e strumenti è stato affrontato nella prima parte del tirocinio. Questo periodo di formazione è durato circa due settimane in cui il tutor mi ha fornito materiale ed esercitazioni per poter comprendere al meglio il contesto tecnologico.\\

\section{Tecnologie}
\subsection*{Java}
Java è un linguaggio di programmazione ad alto livello, orientato agli oggetti e fortemente tipizzato, sviluppato originariamente da Sun Microsystems.\\
La sua popolarità deriva dalla sua portabilità, dalla vasta comunità di sviluppatori e dalle numerose risorse disponibili per l'apprendimento e lo sviluppo.\\
All'interno del mio progetto è stato utilizzato per lo sviluppo del lato back-end del prodotto. 
\\

\subsection*{SQL}
SQL, acronimo di Structured Query Language, è un linguaggio di programmazione utilizzato per gestire e manipolare dati in un database relazionale. SQL fornisce una serie di comandi standardizzati che consentono agli sviluppatori e agli amministratori di database di eseguire operazioni come l'interrogazione dei dati, l'aggiornamento dei dati, l'inserimento di nuovi dati e la creazione e gestione degli schemi dei database.\\
Questo noto linguaggio è stato utilizzato nel progetto per i seguenti motivi:
\begin{itemize}
\item creare le tabelle o trigger utili alla fruizione dei servizi del progetto;
\item eseguire query per inserire, recuperare, eliminare o modificare dati in base alle richieste.
\end{itemize}

\subsection*{Microsoft SQL Server}
SQL Server è un DBMS (Database Management System) relazionale sviluppato da Microsoft. È una piattaforma dati che si utilizza per creare e gestire database, principalmente in ambito aziendale.\\

\subsection*{Spring}
Spring è un framework\textsubscript{g} di sviluppo di applicazioni Java che offre un'ampia gamma di strumenti e librerie per semplificare la creazione di applicazioni aziendali robuste, scalabili e di alta qualità. 
Spring fornisce anche moduli specifici per la gestione dei dati, lo sviluppo web e la sicurezza, rendendolo uno degli strumenti più utilizzati per lo sviluppo Java.\\
A questo framework sono associati tanti altri progetti, che hanno nomi composti come Spring Boot, Spring Data e molti altri.\\
All'interno del progetto Spring è stato utilizzato per sviluppare l'API REST.
\\

\subsection*{Spring Boot}
Spring Boot è un modulo del framework di sviluppo Java Spring che semplifica la creazione, la configurazione e l'avvio di applicazioni Java. Fornisce un ambiente pronto all'uso per sviluppare rapidamente applicazioni Spring, eliminando gran parte della complessità associata alla configurazione. Spring Boot utilizza convenzioni intelligenti e configurazioni predefinite per accelerare lo sviluppo, consentendo agli sviluppatori di concentrarsi sulle funzionalità dell'applicazione anziché sulla configurazione di base.
\\

\subsection*{Spring Data}
Spring Data è un modulo del framework Spring che fornisce un'astrazione per semplificare l'accesso e la gestione dei dati nelle applicazioni Java. Esso offre un'interfaccia unificata per interagire con una varietà di fonti di dati, tra cui database relazionali, database NoSQL e altri servizi di memorizzazione dati.\\
Questo modulo è stato utile nel progetto per interfacciare l'applicazione con il database.
\\

\subsection*{JSON}
JSON, acronimo di JavaScript Object Notation, è un formato leggero di scambio di dati utilizzato comunemente per rappresentare oggetti e strutture di dati.
JSON è ampiamente utilizzato per rappresentare dati in applicazioni web, servizi API, scambio di dati tra client e server, configurazioni di applicazioni e molto altro.\\
All'interno del progetto permette lo scambio di dati tra il lato front-end ed il lato back-end.
\\

\section{Strumenti}
\subsection{Strumenti di sviluppo}
Gli strumenti di sviluppo sono utilizzati direttamente per creare, implementare e testare le funzionalità dell'applicazione. Essi contribuiscono alla realizzazione delle funzionalità dell'applicazione stessa.\\

\subsubsection*{IntelliJ IDEA}
IntelliJ IDEA è un potente ambiente di sviluppo integrato (IDE) sviluppato da JetBrains, progettato principalmente per la programmazione in linguaggi come Java, Kotlin, Scala e altri.
È noto per la sua ricca serie di funzionalità progettate per migliorare l'efficienza degli sviluppatori e semplificare il processo di sviluppo del software.\\
Il seguente IDE è stato utilizzato per la scrittura del codice in Java.\\

\subsubsection*{Maven}
Maven è uno strumento di gestione delle build e delle dipendenze che fornisce un sistema di automazione per la compilazione, il testing, il packaging delle applicazioni e il download delle dipendenze.\\
All'interno di questo progetto Maven è stato utilizzato con Spring Boot per semplificare la gestione delle dipendenze e delle versioni.
\\


\subsubsection*{DBeaver}
DBeaver è un'applicazione di amministrazione di database universale e strumento client SQL. È utilizzato per connettersi, esplorare, gestire e interrogare diversi tipi di database.\\
All'interno del progetto è stato utilizzato per interagire col database.
\\


\subsubsection*{Hibernate}
Hibernate è un framework di mapping\textsubscript{g} oggetto-relazionale (ORM). L'obiettivo principale di Hibernate è semplificare la gestione e l'accesso ai dati in un database relazionale utilizzando oggetti Java anziché scrivere query SQL manualmente.
\\


\subsubsection*{Postman}
Postman è un'applicazione di sviluppo di API (Application Programming Interface) che consente agli sviluppatori di creare, testare, documentare e monitorare le API.\\
Nel corso del progetto è stato uno strumento altamente utilizzato sia come API testing tool.\\

\subsubsection*{JUnit} 
JUnit è un framework di testing per Java utilizzato per la scrittura e l’esecuzione di test d’unità. Questo framework fornisce un ambiente di testing in cui è possibile definire e strutturare test.\\ Questo framework offre funzionalità come annotazioni e vari metodi che semplificano il processo di scrittura di questi test, automatizzando la verifica che il codice soddisfi i requisiti e produca risultati attesi.\\

\subsubsection*{Mockito}
Mockito è un framework di testing per Java che si concentra sulla creazione di oggetti simulati (mock) per testare unità di codice. Gli oggetti mock imitano il comportamento di oggetti reali, consentendo ai test di focalizzarsi su parti specifiche del codice, permettendo ai test di non dover interagire con database o sistemi esterni. Permette inoltre di verificare interazioni con i mock e molto altro, al fine di agevolare il processo di testing.\\

\subsection{Strumenti di supporto}
Gli strumenti di supporto sono utilizzati per attività che sostengono lo sviluppo del progetto. Essi contribuiscono a migliorare la gestione, la qualità e l'efficienza del processo di sviluppo.\\
 
\subsubsection*{Microsoft Teams}
Microsoft Teams è una piattaforma di comunicazione e collaborazione aziendale sviluppata da Microsoft. Offre strumenti per la chat, le videoconferenze, la condivisione di documenti e la gestione dei progetti, consentendo ai team di lavorare insieme in modo efficace sia in ufficio che a distanza.\\
Questa piattaforma è stata utilizzata nel corso del progetto per poter comunicare con il tutor anche quando non era in ufficio o con altri colleghi per determinate situazioni lavorative.


\subsubsection*{Notion}
Notion è un'applicazione di gestione delle informazioni, utilizzata per prendere appunti, creare elenchi di attività, scrivere documenti e molto altro grazie alla sua interfaccia flessibile personalizzabile dagli utenti.\\
Questa applicazione è stata utilizzata nel corso della mia attività di Stage per prendere appunti o tenere traccia delle tasks che dovevo svolgere.\\ 


\subsubsection*{Microsoft Excel}
Microsoft Excel è un'applicazione software di fogli di calcolo sviluppata da Microsoft. È utilizzata per creare, organizzare e analizzare dati in forma di tavole e grafici, offrendo inoltre molte funzionalità per eseguire calcoli.\\
Questa applicazione è stata utilizzata nel progetto allo scopo di velocizzare la creazione di dati fittizi per popolare le tabelle del database per poter testare quanto prodotto.\\


\subsubsection*{Visual Studio Code}
Visual Studio Code è un editor di codice sorgente sviluppato da Microsoft. È progettato per fornire un ambiente di sviluppo leggero, flessibile e personalizzabile per programmatori e sviluppatori.\\
È stato utilizzato nel progetto scaricando varie estensioni per visualizzare l'API e l'UML del database.\\

\subsubsection*{Docker}
Docker è una piattaforma di containerizzazione\textsubscript{g} che consente di creare, distribuire e gestire container\textsubscript{g} virtualizzati\textsubscript{g}. Permette di far funzionare le applicazioni in altri ambienti con facilità.\\
Nel progetto è stato utilizzato per creare un server locale con un'immagine del database MSSQL.\\

\subsubsection*{Swagger}
Swagger è un insieme di strumenti e specifiche che consentono la documentazione, la progettazione e il test di API. Permette una migliore comunicazione dei propri endpoint dell'API semplificandone la lettura e la comprensione.\\

\subsubsection*{Git}
Git è un sistema di controllo versione distribuito utilizzato nello sviluppo software. Consente di tenere traccia delle modifiche apportate al codice sorgente e di semplificare la collaborazione tra sviluppatori in progetti di programmazione.\\


\subsubsection*{GitLab}
GitLab è una piattaforma di sviluppo software basata su web che offre una serie di strumenti per la gestione del ciclo di vita dello sviluppo delle applicazioni. Le principali funzionalità sono: la gestione dei repository Git, la collaborazione tra i membri del team e il monitoraggio delle issue.\\
All'interno del progetto è stato utilizzata come repository per tenere salvato il progresso del progetto. Nella mia attività di stage ho lavorato su un branch a me dedicatomi.\\

\subsubsection*{Git Extensions}
Git Extensions è un'interfaccia grafica per il sistema di controllo versione distribuito Git. Questo software fornisce una modalità visuale per interagire con i repository Git.



