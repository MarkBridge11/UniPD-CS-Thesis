\chapter{Introduzione}
\label{cap:introduzione}

Introduzione al contesto applicativo.\\

\noindent Esempio di utilizzo di un termine nel glossario \\
\gls{api}. \\

\noindent Esempio di citazione in linea \\
\cite{site:agile-manifesto}. \\

\noindent Esempio di citazione nel pie' di pagina \\
citazione\footcite{womak:lean-thinking} \\

\section{Il progetto}

Introduzione all'idea dello stage.

\section{L'azienda}

L’azienda Omicron Consulting è specializzata nello sviluppo di software gestionali e di revisione di processi aziendali. Essa è presente nel mercato ICT dal 1980, spiccando su vari settori, in cui hanno effettuato importanti implementazioni in area ICT come: Manufacturing, Automotive, Aerospace, Logistics e altre aree.
Con particolare riferimento al settore Manufacturing si sono specializzati nello sviluppo di progetti di trasformazione ERP (acquisendo la certificazione VAR di SAP), stringendo alleanze strategiche con realtà ICT nazionali ed internazionali.\\
I servizi per i loro clienti includono anche gestione AMS di sistemi ERP, la realizzazione di progetti sia in ambito di sistemi ERP di Business Intelligence che di sviluppi su sistemi custom.\\
Per completare il pacchetto dei servizi offerti, Omicron ha un’esperienza di alto livello negli ambiti Banking,Finance and Insurance, lavorando con le principali istituzioni bancarie e assicurative italiane.\\
Omicron oltre alle offerte che dedica ai clienti, si occupa anche di garantire un’alta formazione delle proprie risorse, investendo su progetti di ricerca e sviluppo.
\begin{figure}[!h] 
    \centering 
    \includegraphics[width=0.9\columnwidth]{Omicron_Logo} 
    \caption{Logo Omicron}
\end{figure}


\section{Organizzazione del testo}

\begin{description}
    \item[{\hyperref[cap:processi-metodologie]{Il secondo capitolo}}] descrive ...
    
    \item[{\hyperref[cap:descrizione-stage]{Il terzo capitolo}}] approfondisce ...
    
    \item[{\hyperref[cap:analisi-requisiti]{Il quarto capitolo}}] approfondisce ...
    
    \item[{\hyperref[cap:progettazione-codifica]{Il quinto capitolo}}] approfondisce ...
    
    \item[{\hyperref[cap:verifica-validazione]{Il sesto capitolo}}] approfondisce ...
    
    \item[{\hyperref[cap:conclusioni]{Nel settimo capitolo}}] descrive ...
\end{description}

Riguardo la stesura del testo, relativamente al documento sono state adottate le seguenti convenzioni tipografiche:
\begin{itemize}
	\item gli acronimi, le abbreviazioni e i termini ambigui o di uso non comune menzionati vengono definiti nel glossario, situato alla fine del presente documento;
	\item per la prima occorrenza dei termini riportati nel glossario viene utilizzata la seguente nomenclatura: \emph{parola}\glsfirstoccur;
	\item i termini in lingua straniera o facenti parti del gergo tecnico sono evidenziati con il carattere \emph{corsivo}.
\end{itemize}
