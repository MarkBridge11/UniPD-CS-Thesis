\chapter{Introduzione}
\label{cap:introduzione}
\section{Convenzioni tipografiche nel documento}
Durante la stesura del testo sono state adottate le seguenti convenzioni:
\begin{itemize}
\item per ogni elenco puntato viene inserito un punto e virgola (;) alla fine di ogni elemento ed un punto (.) per l'ultimo elemento di ogni elenco;
\item la sezione del glossario contiene i termini ritenuti ambigui o non di uso comune, che necessitano quindi di una loro definizione. Il suo scopo è quello di fornire una comprensione comune del linguaggio utilizzato e di evitare confusione o interpretazioni errate;
\item per la prima occorrenza di un termine inserito nel glossario viene utilizzata la seguente nomenclatura: termine\textsubscript{g}.\\
\end{itemize}

\section{L'azienda ospitante}

\begin{figure}[!h] 
    \centering 
    \includegraphics[width=0.7\columnwidth]{Omicron_Logo} 
    \caption{Logo Omicron}
\end{figure}

\noindent L’azienda Omicron Consulting è specializzata nello sviluppo di software gestionali e di revisione di processi aziendali. Essa è presente nel mercato ICT dal 1980, spiccando su vari settori, in cui hanno effettuato importanti implementazioni in area ICT come: Manufacturing, Automotive, Aerospace, Logistics e altre aree.
Con particolare riferimento al settore Manufacturing si sono specializzati nello sviluppo di progetti di trasformazione ERP (acquisendo la certificazione VAR di SAP), stringendo alleanze strategiche con realtà ICT nazionali ed internazionali.\\
Offrono servizi di gestione e supporto di sistemi ERP, sviluppo di progetti di Business Intelligence e personalizzazione di sistemi software.\\
Per completare il pacchetto dei servizi offerti, Omicron ha un'esperienza di alto livello negli ambiti Banking,Finance and Insurance, lavorando con le principali istituzioni bancarie e assicurative italiane.\\
Omicron oltre alle offerte che dedica ai clienti, si occupa anche di garantire un'alta formazione delle proprie risorse, investendo su progetti di ricerca e sviluppo.

\section{Il progetto}
\subsection{Presentazione}
L'applicativo permette la creazione di richieste di pianificazioni di risorse aziendali per svolgere un determinato incarico. In questo progetto per risorse aziendali si fa sempre riferimento alle risorse umane, quindi al personale o alla forza lavoro dell'azienda.\\
Il Project Manager\textsubscript{g} potrà effettuare richieste di pianificazione di risorse aziendali, chiedendo disponibilità di figure professionali con determinate caratteristiche.\\ 
In seguito ad una richiesta accettata, il Program Manager\textsubscript{g} distribuirà le risorse più adeguate alla richiesta, creando una pianificazione per ogni risorsa richiesta, specificando parametri quali la durata, l'attività da svolgere e molti altri.\\

\subsection{Motivazione del progetto}
L'approccio adottato per la gestione delle pianificazioni delle risorse e la loro disponibilità veniva gestito attraverso fogli Excel compilati e rivisti dai Program manager. Le nuove richieste di pianificazioni vengono comunicate ai Program Manager tramite posta elettronica, telefono e chat, rendendo arduo tenere traccia di tutto.\\
Il progetto nasce dunque dall’esigenza di semplificare la gestione delle richieste e delle pianificazioni, garantendo una visione più rapida della disponibilità delle risorse.\\


\subsection{Il mio ruolo nel progetto}
Le funzionalità da me sviluppate sono due: le richieste e le pianificazioni di risorse aziendali.\\
Il tutto è stato realizzato creando nuove tabelle da inserire nel database per la fruizione dei servizi dell'applicativo e lo sviluppo dell'API\textsubscript{g} REST\textsubscript{g}. Gli endpoint\textsubscript{g} dell'API permettono operazioni CRUD\textsubscript{g} su richieste, pianificazioni e milestone.\\


\section{Organizzazione del testo}
Questa sezione è dedicata alla spiegazione della struttura del documento, per dare indicazioni su come è organizzato il testo.
\begin{description}

\item {\hyperref[cap:introduzione]{Capitolo 1:}} introduce il progetto, il mio ruolo all'interno del progetto e il profilo aziendale. Questo capitolo è l'unica parte in cui si parlerà del progetto nella sua totalità rispetto ai capitoli successivi che saranno inerenti esclusivamente al mio lavoro svolto;

    \item[{\hyperref[cap:analisi-requisiti]{Il secondo capitolo}}] descrive i casi d'uso individuati ed i relativi requisiti;
    
    \item[{\hyperref[cap:descrizione-stage]{Il terzo capitolo}}] approfondisce ...
    
    \item[{\hyperref[cap:analisi-requisiti]{Il quarto capitolo}}] approfondisce ...
    
    \item[{\hyperref[cap:progettazione-codifica]{Il quinto capitolo}}] approfondisce ...
    
    \item[{\hyperref[cap:verifica-validazione]{Il sesto capitolo}}] approfondisce ...
    
    \item[{\hyperref[cap:conclusioni]{Nel settimo capitolo}}] descrive ...
    
\end{description}

