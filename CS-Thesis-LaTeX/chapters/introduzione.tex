\chapter{Introduzione}
\label{cap:introduzione}
\section{Il progetto}
Il progetto OmiPLAN nasce dall’esigenza di gestire le pianificazioni delle risorse aziendali per determinate tasks. Il progetto permette la creazione di richieste di pianificazione o direttamente di pianificazioni per prenotare delle persone per un determinato lasso di tempo a svolgere un incarico.\\
Il Project Manager\textsubscript{g} potrà effettuare richieste di pianificazione di risorse aziendali, chiedendo disponibilità di figure professionali con determinate caratteristiche. Esso potrà inoltre visualizzare tramite una sezione apposita, lo stato delle proprie richieste.\\
In seguito ad una richiesta emessa, il Program Manager distribuirà le risorse più adeguate alla richiesta. Successivamente all'accettazione e alla distribuzione delle risorse, il Project Manager potrà creare una pianificazione per ogni risorsa richiesta, specificando parametri quali la durata e l'attività da svolgere.\\
In alternativa è possibile effettuare delle pianificazioni più dirette, per tasks più brevi, inserendo una delle risorse disponibili attualmente, su un determinato compito. Un esempio di quest’ultima applicazione può essere l’assegnazione per 2 giorni di uno sviluppatore per risolvere un bug.  
Le informazioni relative alle risorse aziendali verranno recuperate da un database.\\
L’applicativo permette la visione della lista completa dei dipendenti o consulenti, con una linea del tempo alla loro destra, contenente le tasks in cui sono coinvolti. È possibile interagire con quest'ultime per avere una view rapida della task o di espanderla o modificarla.\\ 
Infine è possibile utilizzare un calendario per determinare se i giorni di assegnazione sono lavorativi o festivi.


\section{L'azienda}

\begin{figure}[!h] 
    \centering 
    \includegraphics[width=0.7\columnwidth]{Omicron_Logo} 
    \caption{Logo Omicron}
\end{figure}

L’azienda Omicron Consulting è specializzata nello sviluppo di software gestionali e di revisione di processi aziendali. Essa è presente nel mercato ICT dal 1980, spiccando su vari settori, in cui hanno effettuato importanti implementazioni in area ICT come: Manufacturing, Automotive, Aerospace, Logistics e altre aree.
Con particolare riferimento al settore Manufacturing si sono specializzati nello sviluppo di progetti di trasformazione ERP (acquisendo la certificazione VAR di SAP), stringendo alleanze strategiche con realtà ICT nazionali ed internazionali.\\
I servizi per i loro clienti includono anche gestione AMS di sistemi ERP, la realizzazione di progetti sia in ambito di sistemi ERP di Business Intelligence che di sviluppi su sistemi custom.\\
Per completare il pacchetto dei servizi offerti, Omicron ha un’esperienza di alto livello negli ambiti Banking,Finance and Insurance, lavorando con le principali istituzioni bancarie e assicurative italiane.\\
Omicron oltre alle offerte che dedica ai clienti, si occupa anche di garantire un’alta formazione delle proprie risorse, investendo su progetti di ricerca e sviluppo.


\section{Organizzazione del testo}

\begin{description}
    \item[{\hyperref[cap:processi-metodologie]{Il secondo capitolo}}] descrive ...
    
    \item[{\hyperref[cap:descrizione-stage]{Il terzo capitolo}}] approfondisce ...
    
    \item[{\hyperref[cap:analisi-requisiti]{Il quarto capitolo}}] approfondisce ...
    
    \item[{\hyperref[cap:progettazione-codifica]{Il quinto capitolo}}] approfondisce ...
    
    \item[{\hyperref[cap:verifica-validazione]{Il sesto capitolo}}] approfondisce ...
    
    \item[{\hyperref[cap:conclusioni]{Nel settimo capitolo}}] descrive ...
\end{description}

Riguardo la stesura del testo, relativamente al documento sono state adottate le seguenti convenzioni tipografiche:
\begin{itemize}
	\item gli acronimi, le abbreviazioni e i termini ambigui o di uso non comune menzionati vengono definiti nel glossario, situato alla fine del presente documento;
	\item per la prima occorrenza dei termini riportati nel glossario viene utilizzata la seguente nomenclatura: \emph{parola}\glsfirstoccur;
	\item i termini in lingua straniera o facenti parti del gergo tecnico sono evidenziati con il carattere \emph{corsivo}.
\end{itemize}
