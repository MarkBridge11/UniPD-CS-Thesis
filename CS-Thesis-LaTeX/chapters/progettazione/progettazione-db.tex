\section{Progettazione del database}
\subsection{Configurazione di Docker}
I container Docker offrono un'isolamento completo dell'ambiente, che aiuta a evitare conflitti di dipendenze e interferenze con altre applicazioni o servizi che potrebbero essere presenti sul sistema. Per questo motivo si è deciso di utilizzare l'approccio di sandboxing\textsubscript{g} che offre Docker, poichè l'utilizzo di container consente di isolare le applicazioni e i servizi in ambienti virtualizzati, condividendo il kernel del sistema operativo host ma separando le loro risorse e i loro processi.\\
Dato che le tabelle di mia creazione dovevano integrarsi con il database aziendale, tramite il tool Docker Compose\textsubscript{g} e un file di configurazione \textit{docker-compose.yaml}, è stato possibile avviare un nuovo container contenente un'immagine di Microsoft SQL Server, che forniva un backup del database aziendale con dati e tabelle.\\

\subsection{Modello dati}
\textbf{IMMAGINE UML SENZA ATTRIBUTI}\\

\noindent Nel seguente disegno si può notare il database su cui poggia l'API.\\
È stata inserita una nomenclatura <<esterna>> per indicare le tabelle proveniente dal database aziendale.\\
Di seguito elenco le tabelle da me create, spiegandone l'utilizzo e gli attributi:\\

\noindent \textbf{TUTTE LE TABELLE MIE}\\

