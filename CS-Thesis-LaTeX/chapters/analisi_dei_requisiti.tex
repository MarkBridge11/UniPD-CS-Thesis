\chapter{Analisi dei Requisiti}
\label{cap:analisi-requisiti}
\section{Scopo del capitolo}
La seguente sezione di Analisi dei Requisiti rappresenta una dettagliata e approfondita esplorazione delle necessità e delle aspettative che guidano la creazione e lo sviluppo del progetto in questione. Questa analisi è stata condotta al fine di definire chiaramente gli obiettivi e le funzionalità del prodotto, fornendo una base solida per la progettazione e l'implementazione del software.

\section{Descrizione generale}
Ogni caso d'uso è stato schematizzato secondo i seguenti punti:
\begin{itemize}
\item \textbf{attore coinvolto:} in cui si specifica l'attore;
\item \textbf{descrizione:} offre una spiegazione più dettagliata del caso d'uso; 
\item \textbf{precondizioni:} rappresenta la condizione che deve essere soddisfatta e verificata affinchè il caso d'uso possa essere eseguito con successo;
\item \textbf{postcondizioni:} rappresenta lo stato dell'attore in seguito all'esecuzione con successo del caso d'uso;
\item \textbf{estensioni:} in cui si specificano le eventuali estensioni collegate.
\end{itemize}
Vengono inserite anche delle immagini dell'UML\textsubscript{g} per fornire una spiegazione visiva che può aiutare maggiormente la comprensione.

\section{Attori}
...


\section{Casi d'uso}
\subsection{Primo caso}


\section{Tracciamento dei requisiti}


 

